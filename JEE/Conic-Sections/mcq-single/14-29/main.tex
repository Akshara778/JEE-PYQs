\iffalse
  \title{CONIC SECTIONS}
  \author{J.KEDARANANDA}
  \section{mcq-single}
\fi
\item If a $>$ 2b $>$ 0 then the positive value of m for which       $y=mx-b\sqrt{1+m^{2}} $ is a common tangent to $x^{2} + y^{2} = b^{2} $and  $(x-a)^{2} + y^{2} = b^{2}$ is   \hfill {(2002S)}
\begin{multicols}{2}
\begin{enumerate}
    \item $\frac{2b}{\sqrt{a^{2}-4b^{2}}}$\\\\
    \item $\frac{2b}{a-2b}$
    \item $\frac{\sqrt{a^{2}-4b^{2}}}{2b}$\\\\
    \item $\frac{b}{a-2b}$
\end{enumerate} 
\end{multicols}
\item The locus of the mid-point of the line segment joining the focus to a moving point on the parabola $y^{2} = 4ax$ is another parabola with directrix \hfill{(2002S)}
\begin{multicols}{4}
 \begin{enumerate}
    \item x=-a
    \item x=-a/2
    \item x=a
    \item x=a/2
 \end{enumerate}
\end{multicols}
\item The equation of the common tangent to the curves $y^{2}=8x$ and $xy=-1$ is \hfill{(2002S)}
\begin{multicols}{2}
\begin{enumerate}
    \item $3y=9x+2$\\\\
    \item $y=2x+1$
    \item $2y=x+8$\\\\
    \item $y=x+2$
\end{enumerate}
\end{multicols}
\item The area of the quadrilateral formed by the tangents at the end points of the latus rectum to the ellipse $\frac{x^{2}}{9}+\frac{y^{2}}{5}=1$, is \hfill{(2003S)}
\begin{multicols}{2}
 \begin{enumerate}
    \item 27/4 sq.units\\\\
    \item 9 sq.units
    \item 27/2 sq.units\\\\
    \item 27 sq.units
 \end{enumerate}
\end{multicols}
\item The focal chord to $y^{2}=16x$ is tangent to $(x-6)^{2}+y^{2}=2$,then the possible values of the slope of this chord,are \hfill{(2003S)}
\begin{multicols}{2}
\begin{enumerate}
    \item ${-1,1}$\\\\
    \item ${-2,2}$
    \item ${-2,-1/2}$\\\\
    \item ${2,-1/2}$
\end{enumerate}
\end{multicols}{2}
\item For hyperbola $\frac{x^{2}}{\cos^{2}\alpha}-\frac{y^{2}}{\sin^{2}\alpha}=1$ which of the following remains constant with change in '$\alpha$'

\hfill{(2003S)}
\begin{multicols}{2}
\begin{enumerate}
    \item abscissae of vertices\\\\
    \item abscissae of foci
    \item eccentricity\\\\
    \item directrix
\end{enumerate}
\end{multicols}{2}
\item If tangents are drawn to ellipse $x^{2}+2y^{2}=2$,then the locus of the mid-point of the intercept made by the tangents between the coordinate axes is 

\hfill{(2004S)}
\begin{multicols}{2}
\begin{enumerate}
    \item $\frac{1}{2x^{2}}+\frac{1}{4y^{2}}$ \\\\
    \item $\frac{1}{4x^{2}}+\frac{1}{2x^{2}}$ 
    \item $\frac{x^{2}}{2}+\frac{y^{2}}{4}=1$ \\\\
    \item $\frac{x^{2}}{4}+\frac{y^{2}}{2}=1$ 
\end{enumerate}
\end{multicols}
\item The angle between the tangents drawn from the point ${(1,4)}$ to the parabola $y^{2}=4x$ is 

\hfill{(2004S)}
\begin{multicols}{4}
\begin{enumerate}
    \item $\pi/6$ 
    \item $\pi/4$ 
    \item $\pi/3$
    \item $\pi/2$
\end{enumerate}
\end{multicols}
\item If the line $2x+\sqrt{6}y=2$ touches the hyperbola $x^{2}-2y^{2}=4$,then the point of contact is \hfill{(2004S)}
\begin{multicols}{2}
\begin{enumerate}
    \item ${(-2,\sqrt{6})}$\\\\
    \item ${(-5,2\sqrt{6})}$
    \item ${(\frac{1}{2},\frac{1}{\sqrt{6}})}$\\\\
    \item ${(4,-\sqrt{6})}$
\end{enumerate}
\end{multicols}
\item The minimum area of the triangle formed by the tangent to the $\frac{x^{2}}{a^{2}}+\frac{y^{2}}{b^{2}}=1$ \& coordinate axes is \hfill{(2005S)}
\begin{multicols}{2}
\begin{enumerate}
    \item ab sq. units\\\\
    \item $\frac{a^{2}+b^{2}}{2}$ sq. units
    \item $\frac{(a+b)^{2}}{2}$ sq. units\\\\
    \item $\frac{a^{2}+ab+b^{2}}{3}$ sq. units
\end{enumerate}
\end{multicols}
\item Tangent to the curve $y=x^{2}+6$ at a point ${(1,7)}$ touches the circle $x^{2}+y^{2}+16x+12y+c=0$ at a point Q.Then the coordinates of Q are \hfill{(2005S)}
\begin{multicols}{2}
\begin{enumerate}
    \item${(-6,-11)}$\\\\
    \item${(-9,-13)}$
    \item${(-10,-15)}$\\\\
    \item${(-6,-7)}$
\end{enumerate}
\end{multicols}
\item The axis of the parabola is along the line y=x and the distance of its vertex and focus from  origin are $\sqrt2$ and $2\sqrt2$  respectively.If the vertex and focus both lie in the first quadrant,then the equation of the parabola is \hfill{(2006-3M,-1)}
\begin{multicols}{2}
\begin{enumerate}
    \item $(x+y)^{2}=(x-y-2)$\\\\
    \item $(x-y)^{2}=(x+y-2)$
    \item $(x-y)^{2}=4(x+y-2)$\\\\
    \item $(x-y)^{2}=8(x+y-2)$
\end{enumerate}
\end{multicols}
\item A hyperbola, having the transverse axis of length $2\sin\theta$, is confocal with the ellipse $3x^{2}+4y^{2}=12$. Then its equation is \hfill{(2007-3 marks)}
\begin{enumerate}
    \item ${x^{2}\cosec^{2}\theta}-{y^{2}\sec^{2}\theta=1}$\\
    \item $x^{2}\sec^{2}\theta-y^{2}\cosec^{2}\theta=1$\\
    \item $x^{2}\sin^{2}\theta-y^{2}\cos^{2}\theta=1$\\
    \item $x^{2}\cos^{2}\theta-y^{2}\sin^{2}\theta=1$\\
\end{enumerate}
\item Let a and b be non-zero real numbers.Then,the equation $(ax^{2}+by^{2}+c)(x^{2}-5xy+6y^{2}=0)$ represents \hfill{(2008)}
\begin{enumerate}
    \item four straight lines,when c=0 and a,b are of the same sign.\\
    \item two straight lines and a circle,when a=b,and c is of sign opposite to that of a\\
    \item two straight lines and a hyperbola,when a and b are of the same sign and c is of opposite to that of a\\
    \item a circle and a ellipse,when a and b are of the same sign and c is of sign opposite to that of a\\
\end{enumerate}
\item Consider a branch of the hyperbola\\$x^{2}-2y{2}-2\sqrt{2}x-4\sqrt{2}y-6=0$\\with vertex at a point A.Let B be one of the end points of its latus rectum. If C is the focus of the hyperbola nearest to the point A,then the area of the triangle ABC is \hfill{(2008)}
\begin{multicols}{4}
\begin{enumerate}
    \item $1-\sqrt{\frac{2}{3}}$
    \item $\sqrt{\frac{3}{2}}-1$
    \item $1+\sqrt{\frac{2}{3}}$
    \item $\sqrt{\frac{3}{2}}+1$
\end{enumerate}
\end{multicols}
\item The line passing through the extremity $\vec{A}$ of the major axis and extremity $\vec{B}$ of the minor axis of the ellipse\\$x^{2}+9y^{2}=9$\\meets its auxillary circle at the point $\vec{M}$. Then the area of the triangle with the vertices at $\vec{A}$,$\vec{M}$ and the origin $\vec{O}$ is \hfill{(2009)}
\begin{multicols}{4}
\begin{enumerate}
    \item $\frac{31}{10}$
    \item $\frac{29}{10}$
    \item $\frac{21}{10}$
    \item $\frac{27}{10}$
\end{enumerate}
\end{multicols}
