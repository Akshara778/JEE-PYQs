
\iffalse
  \title{Conic sections}
  \author{Suraj Kolluru}
  \section{subjective}
\fi

%   \begin{enumerate}
\item Let '$d$' be the perpendicular distance from the centre of the ellipse $\frac{x^2}{a^2}+\frac{y^2}{b^2}=1$ to the tangent drawn at a point $\vec{P}$ on the ellipse. If $\vec{F_1}$ and $\vec{F_2}$ are the two $foci$ of the ellipse, then show that $\brak{PF_1-PF_2}^2=4a^2\brak{1-\frac{b^2}{d^2}}$. \hfill\brak{1995- 5 marks}

\item Points $\vec{A}$, $\vec{B}$ and $\vec{C}$ lie on a parabola $y^2=4ax$. The tangents to the parabola at $\vec{A}$, $\vec{B}$ and $\vec{C}$ taken in pairs, intersect at points $\vec{P}$, $\vec{Q}$ and $\vec{R}$. Determine the ratios of the areas of triangles $ABC$ and $PQR$. \hfill\brak{1996- 3 marks}

\item From a point $\vec{A}$ common tangents are drawn to the circle $x^2+y^2=\frac{a^2}{2}$ and the parabola $y^2=4ax$. Find the area of the quadrilateral formed by the common tangents, the chord of contact of the circle, and the chord of contact of the parabola. \hfill\brak{1996- 2 marks}

\item A tangent to the ellipse $x^2+4y^2=4$ meets the ellipse $x^2+2y^2=6$ at $\vec{P}$ and $\vec{Q}$. Prove that the tangents at $\vec{P}$ and $\vec{Q}$ of the ellipse $x^2+2y^2=6$ are at right angles. \hfill\brak{1997- 5 marks}

\item The angle between a pair of tangents drawn from a point $\vec{P}$ to the parabola $y^2=4ax$ is 45\degree. Show that the locus of the point $\vec{P}$ is a hyperbola. \hfill\brak{1998- 8 marks}

\item Consider the family of circles $x^2+y^2=r^2$, $2<r<5$. If in the first quadrant, the common tangent to a circle of this family and the ellipse $4x^2+25y^2=100$ meets the coordinate axes at $\vec{A}$ and $\vec{B}$, then find the equation of the locus of the midpoint of $AB$. \hfill\brak{1999- 10 marks}

\item Find the coordinates of all the points $\vec{P}$ on the ellipse $\frac{x^2}{a^2}+\frac{y^2}{b^2}$=1, for which the area of the triangle $PON$ is maximum, where $\vec{O}$ denotes the origin and $\vec{N}$, the foot of the perpendicular from $\vec{O}$ to the tangent at $\vec{P}$. \hfill\brak{1999- 10 marks}

\item Let $ABC$ be an equilateral triangle inscribed in the circle $x^2+y^2=a^2$. Suppose perpendiculars from $\vec{A}$, $\vec{B}$, $\vec{C}$ to the major axis of the ellipse $\frac{x^2}{a^2}+\frac{y^2}{b^2}$=1, $(a>b)$ meet the ellipse respectively at $\vec{P}$, $\vec{Q}$, $\vec{R}$ such that $\vec{P}$, $\vec{Q}$, $\vec{R}$ lie on the same side of the major axis as $\vec{A}$, $\vec{B}$, $\vec{C}$ respectively. Prove that the normals to the ellipse drawn at the points $\vec{P}$, $\vec{Q}$, and $\vec{R}$ are concurrent. \hfill\brak{2000- 7 marks}

\item Let $C_1$ and $C_2$ be respectively, the parabolas $x^2=y-1$ and $y^2=x-1$. Let $\vec{P}$ be any point on $C_1$ and $\vec{Q}$ be any point on $C_2$. Let $P_1$ and $Q_1$ be the reflections of $\vec{P}$ and $\vec{Q}$ respectively with respect to the line $y=x$. Prove that $P_1$ lies on $C_2$, $Q_1$ lies on $C_1$, and $PQ \geq \text{min}({PP_1, QQ_1})$. Hence or otherwise determine points $P_0$ and $Q_0$ on the parabolas $C_1$ and $C_2$ respectively such that $P_0Q_0 \leq PQ$ for all pairs of points $(\vec{P},\vec{Q})$ with $\vec{P}$ on $C_1$ and $\vec{Q}$ on $C_2$. \hfill\brak{2000- 10 marks}

\item Let $\vec{P}$ be a point on the ellipse $\frac{x^2}{a^2}+\frac{y^2}{b^2}=1$, $0<b<a$. Let the line parallel to the y-axis passing through $\vec{P}$ meet the circle $x^2+y^2=a^2$ at the point $\vec{Q}$ such that $\vec{P}$ and $\vec{Q}$ are on the same side of the x-axis. For two positive real numbers $r$ and $s$, find the locus of the point $\vec{R}$ on $PQ$ such that $PR
= r$ as $\vec{P}$ varies over the ellipse. \hfill\brak{2001- 4 marks}

\item Prove that, in an ellipse, the perpendicular from a focus upon any tangent and the line joining the center of the ellipse to the point of contact meet on the corresponding directrix. \hfill\brak{2002- 5 marks}

\item Normals are drawn from the point $\vec{P}$ with slopes $m_1, m_2, m_3$ to the parabola $y^2=4x$. If the locus of $\vec{P}$ with $m_1m_2=\alpha$ is a part of the parabola itself, then find $\alpha$. \hfill\brak{2003- 4 marks}

\item A tangent is drawn to the parabola $y^2-2y-4x+5=0$ at a point $P$ which cuts the directrix at the point $\vec{Q}$. A point $\vec{R}$ is such that it divides $QP$ externally in the ratio 1:2. Find the locus of the point $\vec{R}$. \hfill\brak{2004 - 4 marks}

\item Tangents are drawn from any point on the hyperbola $\frac{x^2}{9}-\frac{y^2}{4}=1$ to the circle $x^2+y^2=9$. Find the locus of the midpoint of the chord of contact. \hfill\brak{2005 - 4 marks}

\item Find the equation of the common tangent in the 1st quadrant to the circle $x^2+y^2=16$ and the ellipse $\frac{x^2}{25}+\frac{y^2}{4}$=1. Also, find the length of the intercept of the tangent between the coordinate axes. \hfill\brak{2005 - 4 marks} 

    
% \end{enumerate}
