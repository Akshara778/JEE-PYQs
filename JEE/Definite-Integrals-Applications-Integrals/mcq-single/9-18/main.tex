\iffalse
  \title{Assignment}
  \author{AI24BTECH11008-G.Sarvajith}
  \section{mcq-single}
\fi

\item 9. Let  $ S $ be the area of the region enclosed by $y = e^{-x^2} ,$  $y = 0 ,$ $x = 0 ,$ and $ x = 1 $; then \hfill (2012)

\begin{enumerate}
\item[(a)] $ S \geq \frac{1}{e} $
\item[(b)] $S \geq 1 - \frac{1}{e}$
\item[(c)] $ S \leq \frac{1}{4} \brak{1 + \frac{1}{\sqrt{e}}}$
\item[(d)] $S\leq\frac{1}{\sqrt{2}}+\frac{1}{\sqrt{e}}\brak{1 -\frac{1}{\sqrt{2}}} $
\end{enumerate}

\item 10. The option(s) with the values of $a$ and $L$ that \ satisfy the following equation is(are) \hfill (JEE Adv. 2015)
\begin{align*}
    \frac{\int_0^{4\pi} e^t  \brak{\sin^6\brak{at} + \cos^4\brak{at}}  dt}{\int_0^{\pi} e^t  \brak{\sin^6\brak{at} + \cos^4\brak{at}} dt} = L?
\end{align*} \\
\begin{enumerate}
\item  [(a)] $a = 2$, $L = \frac{e^{4\pi} - 1}{e^\pi - 1}$
\item [(b)] $a = 2$, $L = \frac{e^{4\pi} + 1}{e^\pi + 1}$
\item [(c)] $a = 4$, $L = \frac{e^{4\pi} - 1}{e^\pi - 1}$
\item[(d)] $a = 4$, $L = \frac{e^{4\pi} + 1}{e^\pi + 1}$
\end{enumerate}

\item 11. Let $f\brak{x}= 7\tan^8x + 7\tan^6x - 3\tan^4x - 3\tan^2x$ for all $x \in \brak{-\frac{\pi}{2}, \frac{\pi}{2}} $. \\
Then the correct expression(s) is(are)

\hfill  (JEE Adv. 2015)
\begin{enumerate}
\item[(a)] $\int_0^{\pi/4} x f\brak{x}dx = \frac{1}{12}$
\item[(b)]$\int_0^{\pi/4} f\brak{x}dx = 0$
\item[(c)] $\int_0^{\pi/4} x f\brak{x}dx = \frac{1}{6}$
\item [(d)] $\int_0^{\pi/4} f\brak{x}dx = 1$
\end{enumerate}

\item 12. Let $f'\brak{x} = \frac{192x^3}{2 + \sin^4\brak{\pi x}}$ for all $x \in \mathbb{R}$ with \\ $f\brak{\frac{1}{2}} = 0$.
If $m \leq \int_{1/2}^{1} f\brak{x} dx \leq M$, then the possible values of $m$ and $M$ are \hfill (JEE Adv. 2015)
\begin{enumerate}
\item [(a)] $m = 13, M = 24$
\item [(b)] $m = \frac{1}{4}, M = \frac{1}{2}$
\item [(c)] $m = -11, M = 0$
\item [(d)] $m = 1, M = 12$
\end{enumerate}
\item 13. Let
\begin{align*}
f\brak{x} = \lim_{n \to \infty}  \brak{\frac{n^n \brak{x+n} x +\brak{\frac{n}{2}} \cdots \brak{x + \frac{n}{n}}}{n!\brak{ x^2 + n^2} \brak{x^2 + \frac{n^2}{4}} \cdots \brak{x^2 + \frac{n^2}{n^2}}}} ^{\frac{x}{n}},
\end{align*}
Then  for all  $x > 0$. \hfill (JEE Adv. 2016)
\begin{enumerate}
\item[(a)] $f\brak{\frac{1}{2}} \geq f\brak{1}$
\item[(b)] $\brak{\frac{1}{3} }\leq f\brak{\frac{2}{3}}$
\item [(c)] $f'\brak{2} \leq 0$
\item [(d)] $\frac{f'\brak{3}}{f\brak{3}} \geq \frac{f'\brak{2}}{f\brak{2}}$
\end{enumerate}
\item 14. Let $f: \mathbb{R} \to (0,1)$ be a continuous function. Then, which of the following function(s) has (have) the value zero at some point in the interval \brak{0,1}?

\hfill (JEE Adv 2016)
\begin{enumerate}
\item[(a)] $x^9 - f\brak{x}$
\item[(b)] $x - \int_{0}^{\frac{\pi}{2} - x} f\brak{t} \cos t \, dt$
\item[(c)] $e^x - \int_{0}^{x} f\brak{t} \sin t \, dt$
\item[(d)] $f\brak{x} + \int_{0}^{\frac{\pi}{2}} f\brak{t} \sin t \, dt$
\end{enumerate}
\item 15. If $g\brak{x} = \int_{\sin x}^{\sin\brak{2x} \sin^{-1}\brak{t} }, dt$, then

\hfill (JEE Adv 2017)
\begin{enumerate}
\item [(a)] $g'\brak{\frac{\pi}{2}} = -2\pi$
\item [(b)] $g' \brak{-\frac{\pi}{2}} = 2\pi$
\item[(c)] $g' \brak{\frac{\pi}{2}} = 2\pi$
\item[(d)] $g' \brak{-\frac{\pi}{2}} = -2\pi$
\end{enumerate}
\item 16. If the line $sx = \alpha$ divides the area of region
$R = \brak{\brak{x, y} \in \mathbb{R}^2 : x^3 \leq y \leq x, 0 \leq x \leq 1}$
into two equal parts, then \hfill (JEE Adv. 2017)
\begin{enumerate}
\item[(a)] $0 < \alpha \leq \frac{1}{2}$
\item[(b)] $\frac{1}{2} < \alpha < 1$
\item[(c)] $2\alpha^4 - 4\alpha^2 + 1 = 0$
\item[(d)] $\alpha^4 + 4\alpha^2 - 1 = 0$
\end{enumerate}
\item 17. If $I = \sum_{k=1}^{98} \int_{k}^{k+1} \frac{k+1}{x(k+1)} \, dx$, then

\hfill (JEE Adv 2017)
\begin{enumerate}
\item[(a)] $1 > \ln 99$
\item[(b)] $1 < \ln 99$
\item[(c)] $1 < \frac{49}{50}$  
\item [(d)] $1 > \frac{49}{50}$
\end{enumerate}
\item 18. For $a \in \mathbb{R}, \abs{a} \geq 1$, let \hfill (JEE Adv 2019)
\begin{align*}
\lim_{{n \to \infty}} \frac{1 + \sqrt{2} + \cdots + \sqrt{n}}{(an+1)^2 + (an+2)^2 + \cdots + (an+n)^2} = 54
\end{align*}
Then the possible value(s) of  are:
\begin{enumerate}
\item[(a)] $9$
\item[(b)] $7$
\item[(c)] $6$
\item[(d)] $8$
\end{enumerate}

