\iffalse
\title{Definite-Integrals-Applications-Integrals}
\author{AI24BTECH11007 - Desaboina Sri Sathwik}
\section{mcq-multiple}
%\begin{enumerate}
\fi
	\item
		If $\int_{0}^{x}f(t)dt=x+\int_{x}^{1}tf(t)dt$, then the value of $f(1)$ is 

		\hfill{(1998-2 Marks)}
		\begin{multicols}{4}
		\begin{enumerate}
			\item $\frac{1}{2}$
			\item $0$
			\item 1
			\item $\frac{-1}{2}$
		\end{enumerate}
		\end{multicols}
	\item
		Let $f(x)=x-[x]$,for every real number x, where [x] is the integral part of x. Then $\int_{-1}^{1}f(x)dx$ is 

		\hfill{(1998-2 Marks)}
		\begin{multicols}{4}
		\begin{enumerate}
			\item 1
			\item 2
			\item 0
			\item $\frac{1}{2}$
		\end{enumerate}  
			\end{multicols}
	\item 
		For which of the following values of m, is the area of the region bounded by the curve $y=x-x^2$ and the line $y=mx$ equals $\frac{9}{2}$?

		\hfill{(1999-3 Marks)}
		\begin{multicols}{4}
		\begin{enumerate}
			\item -4
			\item -2
			\item 2
			\item 4
		\end{enumerate}
		\end{multicols}
	\item 
		Let $f(x)$ be a non-constant twice differentiable function definied on $(-\infty,\infty)$ such that $f(x)=f(1-x)$ and $f'(\frac{1}{4})=0$. Then,

		\hfill{(2008)}
		\begin{enumerate}
			\item $f"(x)$ vanishes at least twice on $[0,1]$
			\item $f'(\frac{1}{2})=0$
			\item $\int_{\frac{-1}{2}}^{\frac{1}{2}}f(x+\frac{1}{2})\sin xdx=0$
			\item $\int_{0}^{\frac{1}{2}}f(t)e^{\sin \pi{t}}dt=\int_{\frac{1}{2}}^{1}f(1-t)e^{\sin \pi{t}}dt$
		\end{enumerate}
	\item
		Area of the region bounded by the curve $y=e^x$ and lines $x=0$ and $y=e$ is

		\hfill{(2009)}
		\begin{multicols}{2}
		\begin{enumerate}
			\item $e-1$
			\item $\int_{1}^{e}\ln (e+1-y)dy$
			\item $e-\int_{0}^{1}e^xdx$
			\item $\int_{1}^{e}\ln ydy$
		\end{enumerate}
			\end{multicols}
	\item 
		If $I_n=\int_{-\pi}^{\pi}\frac{\sin{nx}}{(1+{\pi}^x)\sin x}dx$ $n=0,1,2, ...,$ then

		\hfill{(2009)}
		\begin{multicols}{2}
		\begin{enumerate}
			\item $I_{n}=I_{n+2}$
			\item $\sum_{m=1}^{10}I_{2m+1}=10\pi$
			\item $\sum_{m=1}^{10}I_{2m}=0$
			\item $I_{n}=I_{n+1}$
		\end{enumerate}
			\end{multicols}
	\item 
		The value(s) of $\int_{0}^{1}\frac{x^4(1-x)^4}{1+x^2}dx$ is(are)

		\hfill{(2010)}
		\begin{multicols}{2}
		\begin{enumerate}
			\item  $\frac{22}{7}-\pi$
			\item $\frac{2}{105}$
			\item $0$
			\item $\frac{71}{15}-\frac{3\pi}{2}$
		\end{enumerate}
			\end{multicols}
	\item 
		Let $f$ be a real-valued function defined on the interval $(0,\infty)$ by $f(x)=\ln x+\int_{0}^{x}\sqrt{1+\sin t}dt$. Then which of the following statement(s) is(are) true?

		\hfill{(2010)}
		\begin{enumerate}
			\item $f"(x)$ exists for all $x\in(0,\infty)$
			\item $f'(x)$ exists for all $x\in(0,\infty)$ and $f'$ is continuous on $(0,\infty)$, but not differentiable on $(0,\infty)$
			\item there exists $\alpha>1$ such that $|f'(x)|<|f(x)|$ for all $x\in(\alpha,\infty)$
			\item there exists $\beta>0$ such that $|f(x)|+|f'(x)|\leq\beta$ for all $x\in(0,\infty)$
		\end{enumerate}







%\end{enumerate}
