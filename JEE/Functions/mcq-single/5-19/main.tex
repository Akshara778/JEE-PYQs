%\begin{document}

\iffalse
\title{10/A/C/5-19}
\author{EE24BTECH11040 - Mandara Hosur}
% \maketitle
% \newpage
% \bigskip
\section{mcq-single}
\fi

%\begin{enumerate}

\item If $f\brak{x} = \cos{\brak{\ln{x}}}$, then $$f\brak{x}f\brak{y}-\frac{1}{2} \sbrak{f\brak{\frac{x}{y}}+f\brak{xy}}$$ has the value
\hfill{\brak{1983 - 1 Mark}}
\begin{multicols}{2}
	\begin{enumerate}
		\item -1 
		\item $\frac{1}{2}$
		\item -2 
		\item none of these
	\end{enumerate}
\end{multicols}

\item The domain of definition of the function
$y = \frac{1}{\log_{10}{\brak{1-x}}} + \sqrt{x+2}$ is
\hfill{\brak{1983 - 1 Mark}}
\begin{multicols}{2}
	\begin{enumerate}
		\item \brak{-3, -2} excluding -2.5 
		\item $\sbrak{0, 1}$ excluding 0.5
		\item $\left[-2, 1\right)$ excluding 0 
		\item none of these
	\end{enumerate}
\end{multicols}

\item Which of the following functions is periodic?
\hfill{\brak{1983 - 1 Mark}}
\begin{enumerate}
\item $f\brak{x}=x-\sbrak{x}$ where $\sbrak{x}$ denotes the largest integer less than or equal to the real number $x$
\item $f\brak{x}=\sin{\frac{1}{x}}$ for $x\neq0$, $f\brak{0}=0$
\item $f\brak{x}=x\cos{x}$
\item none of these
\end{enumerate}

\item Let $f\brak{x}=\sin{x}$ and $g\brak{x}=\ln{|x|}$. If the ranges of the composition functions $fog$ and $gof$ are $R_1$ and $R_2$ respectively, then 
\hfill{\brak{1994 - 2 Marks}}
\begin{enumerate}
\item $R_1=\cbrak{u:-1\le u<1}$, $R_2=\cbrak{v:-\infty<v<0}$
\item $R_1=\cbrak{u:-\infty<u<0}$, $R_2=\cbrak{v:-1\le v\le0}$
\item $R_1=\cbrak{u:-1<u<1}$, $R_2=\cbrak{v:-\infty<v<0}$
\item $R_1=\cbrak{u:-1\le u\le1}$, $R_2=\cbrak{v:-\infty<v\le0}$
\end{enumerate}

\item Let $f\brak{x}=\brak{x+1}^{2}-1$, $x\ge-1$. Then the set $\cbrak{x:f\brak{x}=f^{-1}\brak{x}}$ is
\hfill{\brak{1995}}
\begin{enumerate}
\item $\cbrak{0, -1, \frac{-3+i\sqrt{3}}{2}, \frac{-3-i\sqrt{3}}{2}}$
\item \cbrak{0, 1, -1}
\item \cbrak{0, -1}
\item empty
\end{enumerate}

\item The function $f\brak{x}=|px-q|+r|x|$, $x\in\brak{-\infty,\infty}$ where $p>0$, $q>0$, $r>0$ assumes its minimum value only on one point if
\hfill{\brak{1995}}
\begin{multicols}{2}
	\begin{enumerate}
		\item $p\neq q$
		\item $r\neq q$
		\item $r\neq p$ 
		\item $p=q=r$
	\end{enumerate}
\end{multicols}

\item Let $f\brak{x}$ be defined for all $x>0$ and be continuous. Let $f\brak{x}$ satisfy $f\brak{\frac{x}{y}}=f\brak{x}-f\brak{y}$ for all $x$, $y$ and $f\brak{e}=1$. Then
\hfill{\brak{1995S}}
\begin{multicols}{2}
	\begin{enumerate}
		\item $f\brak{x}$ is bounded 
		\item $f\brak{\frac{1}{x}}\to0$ as $x\to0$
		\item $xf\brak{x}\to1$ as $x\to0$ 
		\item $f\brak{x}=\ln{x}$
	\end{enumerate}
\end{multicols}

\item If the function $f:[1,\infty)\to[1,\infty)$ is defined by $f\brak{x}=2^{x\brak{x-1}}$, then $f^{-1}\brak{x}$ is
\hfill{\brak{1999 - 2 Marks}}
\begin{multicols}{2}
	\begin{enumerate}
		\item $\brak{\frac{1}{2}}^{x\brak{x-1}}$ 
		\item $\frac{1}{2}\brak{1+\sqrt{1+4\log_{2}{x}}}$
		\item $\frac{1}{2}\brak{1-\sqrt{1+4\log_{2}{x}}}$ 
		\item not defined
	\end{enumerate}
\end{multicols}

\item Let $f:R\to R$ be any function. Define $g:R\to R$ by $g\brak{x}=|f\brak{x}|$ for all $x$. Then $g$ is
\hfill{\brak{2000S}}
\begin{enumerate}
\item onto if $f$ is onto
\item one-one if $f$ is one-one
\item continuous if $f$ is continuous
\item differentiable if $f$ is differentiable
\end{enumerate}

\item The domain of definition of the function $f\brak{x}$ given by the equation $2^{x}+2^{y}=2$ is
\hfill{\brak{2000S}}
\begin{multicols}{2}
	\begin{enumerate}
		\item $0<x\le1$ 
		\item $0\le x\le1$
		\item $-\infty<x\le0$ 
		\item $-\infty<x<1$
	\end{enumerate}
\end{multicols}

\item Let $g\brak{x}=1+x-\sbrak{x}$ and
\begin{equation*}
f\brak{x}=
\begin{cases}
-1, & \text{$x<0$} \\
0, & \text{$x=0$.} \\
1, & \text{$x>0$}
\end{cases}
\end{equation*}
Then for all $x$, $f\brak{g\brak{x}}$ is equal to
\hfill{\brak{2001S}}
\begin{multicols}{2}
	\begin{enumerate}
		\item $x$ 
		\item 1
		\item $f\brak{x}$ 
		\item $g\brak{x}$
	\end{enumerate}
\end{multicols}

\item If $f:[1,\infty)\to[2,\infty)$ is given by $f\brak{x}=x+\frac{1}{x}$ then $f^{-1}\brak{x}$ equals
\hfill{\brak{2001S}}
\begin{multicols}{2}
	\begin{enumerate}
		\item $\frac{\brak{x+\sqrt{x^{2}-4}}}{2}$ 
		\item $\frac{x}{\brak{1+x^{2}}}$
		\item $\frac{\brak{x-\sqrt{x^{2}-4}}}{2}$ 
		\item $1+\sqrt{x^{2}-4}$
	\end{enumerate}
\end{multicols}

\item The domain of definition of $f\brak{x}=\frac{\log_{2}{\brak{x+3}}}{x^{2}+3x+2}$ is
\hfill{\brak{2001S}}
\begin{multicols}{2}
	\begin{enumerate}
		\item $R \backslash \cbrak{-1,-2}$ 
		\item $\brak{-2,\infty}$
		\item $R \backslash \cbrak{-1,-2,-3}$ 
		\item $\brak{-3,\infty}\backslash\cbrak{-1,-2}$
	\end{enumerate}
\end{multicols}
                                                                                        
\item Let $E=\cbrak{1,2,3,4}$ and $F=\cbrak{1,2}$. Then the number of onto functions from E to F is
\hfill{\brak{2001S}}
\begin{multicols}{4}
	\begin{enumerate}
		\item 14 
		\item 16 
		\item 12 
		\item 8
	\end{enumerate}
\end{multicols}

\item Let $f\brak{x}=\frac{\alpha x}{x+1}$, $x\neq-1$. Then, for what value of $\alpha$ is $f\brak{\brak{x}}=x$?
\hfill{\brak{2001S}}
\begin{multicols}{4}
	\begin{enumerate}
		\item $\sqrt{2}$ 
		\item $-\sqrt{2}$ 
		\item 1 
		\item -1
	\end{enumerate}
\end{multicols}

%\end{enumerate}

%\end{document}
