\iffalse
\title{Circles}
\author{Eshan sharma}
\section{mcq-single}
\fi

    \item A square is inscribed in the circle $x^{2} + y^{2} - 2x +4y +3= 0.$ Its sides are parellel to the coordinate axes. The one vertex of the square is \hfill {(1980)}
    \begin{multicols}{2}
    	\begin{enumerate}
    		\item $\brak{1+\sqrt{2}, -2}$ 
    		\item $\brak{1-\sqrt{2}, -2}$
    		\item $\brak{1, -2 +\sqrt{2}}$
    		\item none of these
    	\end{enumerate}
    \end{multicols}
    \item Two circles $x^{2} + y^{2} = 6$ and $x^{2} + y^{2}-6x +8=0$ are given. Then the equation of the circle through their points of intersection and the point $\brak{1,1}$ is \hfill {(1980)}
    \begin{enumerate}
    	\item $x^{2}+y^{2}-6x+4=0$ 
    	\item $x^{2}+y^{2}-3x+1=0$
    	\item $x^{2}+y^{2}-4y+2=0$
    	\item none of these
    \end{enumerate}
    \item The centre of the circle passing through the point $\brak{0.1}$ and touching the curve y = $x^{2}$ at \\ $\brak{2,4}$.
    \hfill {(1983 - 1 mark)}
    \begin{multicols}{2}
    	\begin{enumerate}
    		\item $\brak{\frac{-16}{5},\frac{27}{10}}$
    		\item $\brak{\frac{-16}{7},\frac{53}{10}}$
    		\item $\brak{\frac{-16}{5},\frac{53}{10}}$
    		\item none of these
    	\end{enumerate}
    \end{multicols}
    \item The equation of circle passing through $\brak{1,1}$ and points of intersection of the circles $x^{2}+y^{2}+13x-3y=0$ and $2x^{2}+2y^{2}+4x-7y-25=0$ is
    \hfill {(1983 - 1 mark)}
    \begin{enumerate}
    	\item $4x^{2}+4y^{2}-30x-10y-25=0$
    	\item $4x^{2}+4y^{2}+30x-13y-25=0$
    	\item $4x^{2}+4y^{2}-17x-10y+25=0$
    	\item none of these
    \end{enumerate}
    \item The locus of the midpoint of a chord of the circle $x^{2}+y^{2}=4$ which subtends a right angle at the origin is \hfill {(1984 - 2 mark)}
    \begin{enumerate}
    	\item $x+y=2$
    	\item $x^{2}+y^{2}=1$
    	\item $x^{2}+y^{2}=2$
    	\item $x+y=1$
    \end{enumerate}
    \item If a circle is passing through the point $\brak{a,b}$ and it is cutting the circle $x^{2}+y^{2}=k^{2}$ orthogonally, then the equation of the locus of its centre \\ is 
    \hfill {(1988 - 2 mark)}
    \begin{enumerate}
    	\item $2ax + 2by - (a^{2}+b^{2}+k^{2}) = 0$
    	\item $2ax + 2by - (a^{2}-b^{2}+k^{2}) = 0$
    	\item $x^{2} + y^{2}-3ax-4by+ (a^{2}+b^{2}-k^{2}) = 0$
    	\item $x^{2} + y^{2}-2ax-3by+ (a^{2}-b^{2}-k^{2}) = 0$
    \end{enumerate}
    \item If the two circles $(x-1)^{2} + (y-3)^{2} = r^{2}$ and $x^{2}+y^{2}-8x+2y+8=0$ intersect in two distinct points, then \hfill {(1989 - 2 mark)} 
    \begin{enumerate}
    	\item $2<r<8$
    	\item $r<2$
    	\item $r=2$
    	\item $r>2$
    \end{enumerate}
    \item The lines $2x-3y=5$ and $3x-4y=7$ are diameters of a circle of area $154$ sq. units. The equation of this circle is\hfill {(1989 - 2 mark)}
    \begin{enumerate}
    	\item $x^{2}+y^{2}+2x-2y=62$
    	\item $x^{2}+y^{2}+2x-2y=47$
    	\item $x^{2}+y^{2}-2x+2y=47$
    	\item $x^{2}+y^{2}-2x+2y=62$
    \end{enumerate}
    \item The centre of the circle passing through the points $\brak{0,0}$,$\brak{1,0}$ and touching the circle $x^{2}+y^{2}=9$ is
    \hfill {(1992 - 1 mark)}
    \begin{enumerate}
    	\item $\brak{\frac{3}{2},\frac{1}{2}}$
    	\item $\brak{\frac{1}{2},\frac{3}{2}}$
    	\item $\brak{\frac{1}{2},-2\frac{1}{2}}$
    	\item none of these
    \end{enumerate}
    \item The locus of the centre of a circle, which touches the circle is $x^{2}+y^{2}-6x-6y+14=0$ and also touches the y-axis, is given by the equation: \hfill {(1993 - 1 mark)}
    \begin{enumerate}
    	\item $x^{2}-6x-10y+14=0$
    	\item $x^{2}-10x-6y+14=0$
    	\item $y^{2}-6x-10y+14=0$
    	\item $y^{2}-10x-6y+14=0$
    \end{enumerate}
    \item The circles $x^{2}-10x+16=0$ and $x^{2}+y^{2}=r^{2}$ intersect each other in the two distinct points\\ if
    \hfill {(1994)}
    \begin{enumerate}
    	\item $r<2$
    	\item $r>8$
    	\item $2<r<8$
    	\item $2\leq r\leq8$
    \end{enumerate}
    \item The angle between the pair of tangents drawn from the point $\vec{P}$ to the circle $x^{2}+y^{2}+4x-6y+9\sin^{2}{\alpha}+13\cos^{2}{\alpha}=0$ is $2\alpha$. The equation of the locus of the point $\vec{P}$ is
    \hfill {(1996 - 1 mark)}
    \begin{enumerate}
    	\item $x^{2}+y^{2}+4x-6y+4=0$
    	\item $x^{2}+y^{2}+4x-6y-9=0$
    	\item $x^{2}+y^{2}+4x-6y-4=0$
    	\item $x^{2}+y^{2}+4x-6y+9=0$
    \end{enumerate}
    \item If two distinct chords, drawn from the point $\brak{p,q}$ on the circle $x^{2}+y^{2}=px+qy$ (where $pq \neq 0$) are bisected by the x-axis, then which are true
    \hfill {(1999 - 1 mark)}
    \begin{enumerate}
    	\item $p^{2}=q^{2}$
    	\item $p^{2}=8q^{2}$ 
    	\item $p^{2}<8q^{2}$
    	\item $p^{2}>8q^{2}$
    \end{enumerate}
    
%\end{enumerate}
%\end{document}
