\iffalse
\title{Assignment 1 quadratic equation}
\author{AI24BTECH11020 - RISHIKA K}
\section{mcq-single}
\fi

%\begin{enumerate}
\item The equation $e^{\sin x}-e^{-\sin x}-4=0$ has:
	\hfill \brak{2012}
	\begin{enumerate}[label=\alph*.]
	\item  infinite number of real roots
	\item  no real roots
	\item  exactly one real root
	\item  exactly four roots
	\end{enumerate}
\item The real number k for which the equation, $2x^3+3x+4=0$ has two distinct real roots in $\sbrak{0,1}$
        \hfill \brak{JEE M 2013}
	\begin{enumerate}[label=\alph*.]
        \item  lies between 1 and 2
	\item  lies between 2 and 3
	\item  lies between -1 and 0
	\item  does not exist.
	\end{enumerate}
\item The number of values of $k$, for which the system of equations:\\
	$\brak{k+1}x+8y=4k$\\
	$kx+\brak{k+3}y=3k-1$\\
has no solution, is\hfill \brak{JEE M 2013}
        \begin{enumerate}[label=\alph*.]
	\begin{multicols}{2}
	\item   infinite
	\item     1
	\item     2
	\item     3
	\end{multicols}
	\end{enumerate}
\item If the equations $x^2+2x+3=0$ and $ax^2+bx+c=0$,$a,b,c$ $\in$ $\mathbb{R}$ , have a common root , then $a:b:c$ is
	\hfill \brak{JEE M 2013}
	\begin{enumerate}[label=\alph*.]
	\begin{multicols}{2}
	\item  1:2:3
	\item  3:2:1
	\item  1:3:2  
	\item  3:1:2
	\end{multicols}
	\end{enumerate}
\item If $a\in \mathbb{R}$ and the equation  $-3(x-\sbrak{x})^{2}+2(x-\sbrak{x})+a^{2}=0$ (where $\sbrak{x}$ denotes the greatest integer $\leq x$) has no integral solution, then all possible values of $a$ lie in the interval: 
	\hfill \brak{JEE M 2014}
          \begin{enumerate}[label=\alph*.]
	  \begin{multicols}{2}
	  \item $\brak{-2,-1}$
	  \item $\brak{-\infty,2}\cup\brak{2,\infty}$
	  \item $\brak{-1,0}\cup{0,1}$
	  \item $\brak{1,2}$
	  \end{multicols}
	  \end{enumerate}
  \item Let $\alpha$ and $\beta$ be the roots of equation $px^2 + qx +r = 0$, $p\neq0$. If $p,q,r$ are in $A.P$. and $\frac{1}{\alpha}+\frac{1}{\beta}=4$, then the value of $\abs{\alpha-\beta}$ is
	 \hfill \brak{JEE M 2014}
	\begin{enumerate}[label=\alph*.]
	\begin{multicols}{2}
	\item $\frac{\sqrt{34}}{9}$
	\item $\frac{2\sqrt{13}}{9}$
	\item $\frac{\sqrt{61}}{9}$
	\item $\frac{2\sqrt{17}}{9}$
	\end{multicols}
	\end{enumerate}
\item Let $\alpha$ and $\beta$ be the roots of the equation $x^2-6x-2=0$. If $a_n=\alpha^n-\beta^n$, for $n\geq1$, then the value of $\frac{a_{10}-2a_8}{2a_9}$ is equal to:
	\hfill \brak{JEE M 2015}
	\begin{enumerate}[label=\alph*.]
	\begin{multicols}{4}
        \item   3
	\item  -3
	\item   6
	\item  -6
	\end{multicols}
	\end{enumerate}
\item The sum of all real values of $x$ satisfying the equation $\brak{x^2-5x+5}^{x^2+4x+60}=1$ is :
	\hfill \brak{JEE M 2016}
	\begin{enumerate}[label=\alph*.]
	\begin{multicols}{4}
        \item  6
	\item  5
        \item  3
	\item  -4
	\end{multicols}
	\end{enumerate}
\item If $\alpha,\beta\in \mathbb{C}$ are the distinct roots, of the equation $x^2-x+1=0$ , then $\alpha^{101}+\beta^{107}$ is equal to :
	\hfill \brak{JEE M 2018}
	\begin{enumerate}[label=\alph*.]
	\begin{multicols}{4}
	\item  0
	\item  1
	\item  2
	\item  -1
	\end{multicols}
	\end{enumerate}
\item Let $p,q$ $\in$ $\mathbb{R}$. If $2-\sqrt{3}$ is a root of the quadratic equation, $x^2+px+q=0$,then:
	\hfill \brak{JEE M 2019- 9 April(M)}
	\begin{enumerate}[label=\alph*.]
	\begin{multicols}{2}
	\item $p^2-4q+12=0$
	\item $q^2-4p-16=0$
	\item $q^2+4p+14=0$ 
	\item $p^2-4q-12=0$
	\end{multicols}
	\end{enumerate}
%\end{enumerate}


