\iffalse
  \title{Assignment}
  \author{AI24BTECH11022 - Pabbuleti Venkata Charan Teja}
  \section{mains}
\fi

%\begin{enumerate}
\item 
If $\alpha\neq\beta$ but ${\alpha^{2}=5\alpha-3}$ and ${\beta^{2}=5\beta-3}$ then the equation having $\frac{\alpha}{\beta}$ and $\frac{\beta}{\alpha}$ as its roots is\hfill(2002)

\begin{multicols}{2}

\begin{enumerate}

\item$3x^{2}-19x+3=0$

\item$3x^{2}+19x-3=0$

\item $3x^{2}-19x-3=0$

\item $x^{2}-5x+3=0$

\end{enumerate}

\end{multicols}

\item 
Difference between the corresponding roots of ${x^{2}+ax+b=0}$ and ${x^{2}+bx+a=0}$ is same and $a\not=b$, then \hfill(2002)
\begin{multicols}{2}
    
\begin{enumerate}

\item $a+b+4=0$

\item $a+b-4=0$

\item $a-b-4=0$

\item $a-b+4=0$
\end{enumerate}
\end{multicols}

\item 
Product of real roots of the equation ${t^{2}x^{2}+\abs{x}+9=0}$\hfill(2002)
\begin{multicols}{2}
\begin{enumerate}

\item is always positive

\item is always negative

\item does not exist

\item none of these
\end{enumerate}
\end{multicols}

\item 
If $p$ and $q$ are the roots of the equation ${x^{2}+px+q=0}$, then\hfill(2002)
\begin{multicols}{2}
\begin{enumerate}

\item$p=1,q=2$

\item$p=0,q=1$

\item$p=-2,q=0$

\item$p=-2,q=1$
\end{enumerate}
\end{multicols}

\item 
If $a,b,c$ are distinct $+ve$ real numbers and ${a^{2}+b^{2}+c^{2}=1}$ then ${ab+bc+ca}$ is\hfill(2002)
\begin{multicols}{2}
\begin{enumerate}

\item less than 1

\item equal to 1

\item greater than 1

\item any real no.
\end{enumerate}
\end{multicols}
\item 
If the sum of the roots of the quadratic equation ${ax^{2}+bx+c=0}$ is equal to the sum of the squares of their reciprocals, then $\frac{a}{c},\frac{b}{a},\frac{c}{b}$ are in 

\hfill(2003) 

\begin{enumerate}
\item
Arithmetic - Geometric Progression

\item 
Arithmetic Progression

\item 
Geometric Progression

\item 
Harmonic Progression
\end{enumerate}

\item 
The value of '$a$' for which one root of the quadratic equation ${\brak{a^{2}-5a+3}x^{2}+\brak{3a-1}x+2=0}$ is twice as large as the other is \hfill(2003)
\begin{multicols}{2}

\begin{enumerate}
\item $-\frac{1}{3}$

\item $\frac{2}{3}$

\item $-\frac{2}{3}$

\item $\frac{1}{3}$
\end{enumerate}

\end{multicols}

%\end{enumerate}