\iffalse
\title{Assignment-1}
\author{EE24BTECH11048-NITHIN.K}
\section{mcq-multiple}
\fi
%\begin{enumerate}
%1
	\item If $x+\abs{y}=2y$, then y as a function of x is \hfill{(1984-3marks)}
		\begin{enumerate}
			\item defined for all real x
			\item continuous at $x=0$
			\item differentiable for all x
			\item such that $\frac{dy}{dx}=\frac{1}{3}$ for$x<0$ 
		\end{enumerate}
%2
	\item If $f\brak{x}=x\brak{\sqrt{x}-\sqrt{x+1}}$, then- \hfill{(1985-2marks)}
		\begin{enumerate}
			\item $f\brak{x}$ is continuous but not differentiable at $x=0$
			\item $f\brak{x}$ is differentiable at $x=0$
			\item $f\brak{x}$ is not differentiable at $x=0$
			\item none of these 
		\end{enumerate}
%3
	\item The function $f\brak{x}=1+\abs{\sin{x}}$ is \hfill{(1986-2marks)}
		\begin{enumerate}
			\item continuous nowhere
			\item continuous everywhere
			\item differentiable nowhere 
			\item not differentiable at $x=0$
			\item not differentiable at infinite number of points 
		\end{enumerate}
%4
	\item Let $\sbrak{x}$ denote the greatest integer less than or equal to x. If $f\brak{x}=\sbrak{x\sin{\pi x}}$, then $f\brak{x}$ is \hfill{(1986-2marks)}
		\begin{enumerate}
			\item continuous at $x=0$
			\item continuous in $\brak{-1,0}$
			\item differentiable at $x=1$ 
			\item differentiable in $\brak{-1,1}$ 
			\item none of these 
		\end{enumerate}
%5
	\item The set of all points where the function $f\brak{x}=\frac{x}{(1+\abs{x})}$ is differentiable, is \\ \hfill{(1987-2marks)}
		\begin{enumerate}
			\item $\brak{- \infty, \infty}$
			\item $\lsbrak{0}, \rbrak{\infty}$
			\item $\brak{- \infty,0}$ $\bigcup$ $\brak{0, \infty}$
			\item $\brak{0, \infty}$
			\item None 
		\end{enumerate}
%6
	\item The function \\ $f(x)=\begin{cases}\abs{x-3} & \text{,} x \geq 1 \\ \frac{x^2}{4}-\frac{3x}{2}+\frac{13}{4}, & \text{,} x<1 \end{cases}$,is \hfill{(1988-2marks)}
			\begin{enumerate}
				\item continuous at $x=1$
				\item differentiable at $x=1$
				\item continuous at $x=3$
				\item differentiable at $x=3$
			\end{enumerate}
%7
		\item If $f\brak{x} = \frac{1}{2}x-1$, then on the interval $\sbrak{0,\pi}$ \hfill{(1989-2marks)}
			\begin{enumerate}
				\item $\tan{\sbrak{f\brak{x}}}$ and $\frac{1}{f(x)}$ are both continuous
				\item $\tan{\sbrak{f\brak{x}}}$ and $\frac{1}{f(x)}$ are both discontinuous
				\item $\tan{\sbrak{f\brak{x}}}$ and $f^{-1}x$ are both continuous
				\item $\tan{\sbrak{f\brak{x}}}$ is continuous but $\frac{1}{f\brak{x}}$ is not
			\end{enumerate}
%8
		\item The value of $\lim_{x\to0}{\frac{\sqrt{\frac{1}{2}(1-\cos{2x})}}{x}}$ \hfill{(1991-2marks)}
			\begin{enumerate}
				\item 1
				\item -1
				\item 0
				\item none of these 
			\end{enumerate}
%9
		\item The following functions are continuous on $\brak{0,\pi}$ \hfill{(1991-2marks)}
			\begin{enumerate}
				\item $\tan{x}$
				\item $\int_{0}^{x}t\sin{\frac{1}{t}}dt$ 
				\item $\begin{cases} 1 & \text{,} 0<x\leq\frac{3\pi}{4} \\
				2\sin{\frac{2}{9}x} & \text{,} \frac{3\pi}{4}<x<\pi \end{cases}$
			\item $\begin{cases} x\sin{x} & \text{,} 0<x\leq\frac{\pi}{2} \\ \frac{\pi}{2}\sin{\pi+x} & \text{,} \frac{\pi}{2}<x<\pi \end{cases}$
			\end{enumerate}
%10
		\item Let $f\brak{x}=\begin{cases} 0 & \text{,} x<0 \\
                             x^2 & \text{,} x\leq0 \end{cases}$ then for all x \hfill{(1994)}
			     \begin{enumerate}
				     \item $f^{\prime}$ is differentiable
				     \item f is differentiable
				     \item $f^{\prime}$ is continuous
				     \item f is continuous
			     \end{enumerate}
%\end{enumerate}
