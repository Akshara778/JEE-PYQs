\iffalse
\title{11. Limits, Continuity and Differentiability}
\author{EE24BTECH11049 \\ Patnam Shariq Faraz Muhammed}
\section*{mcq-multiple}
\fi

%\begin{enumerate}
    \item 
%1st question
    Let $g\brak{x}=xf\brak{x}$, where $f\brak{x} = \begin{cases}
        x\sin\frac{1}{x}, & x\neq 0\\
        0, & x=0
    \end{cases}$. At $x=0$ 
	
    \hfill(1995)
    
    \begin{enumerate}
      	\item $g$ is differentiable but $g^{\prime}$ is not continuous
        \item $g$ is differentiable while $f$ is not
        \item both $f$ and $g$ are differentiable
	\item $g$ is differentiable and $g^{\prime}$ is continuous 
    \end{enumerate}

    \item 
%2nd question
    The function $f\brak{x} = \text{max}\cbrak{\brak{1-x},\brak{1+x},2}$, $x\in \brak{-\infty,\infty}$  

    \hfill(1995)
    
    \begin{enumerate}    
        \item continuous at all points
        \item differentiable at all points
        \item differentiable at all points except at $x=l$ and $x=-1$
        \item continuous at all points except at $x=l$ and $x=-1$ where it is discontinuous
    \end{enumerate}

    \item 
%3rd question
    Let $h\brak{x}$ = min$\cbrak{x,x^{2}}$ 
    
    \hfill(1998-2M)

    \begin{enumerate}
	\item $h$ is continuous for all $x$
        \item $h$ is differentiable for all $x$
	\item $h^{\prime}\brak{t}=1$, for all $x>1$
        \item $h$ is not differentiable at two values of $x$
    \end{enumerate}

    \item 
%4th question
    $\lim_{x \to1}\frac{\sqrt{1-\cos 2\brak{x-1}}}{x-1}$ 

    \hfill(1998-2M)
    
    \begin{enumerate}
        \item exists and it equals $\sqrt{2}$
        \item exists and it equals $-\sqrt{2}$
        \item does not exist because $x-1\to 0$
        \item does not exist because the left-hand limit is not equal to the right-hand limit
    \end{enumerate}


    \item 
%5th question
    If $f\brak{x}$ = min $\cbrak{1,x^2,x^3}$ 

    \hfill(2006, 5M, -1)
    
    \begin{enumerate}
        \item $f\brak{x}$ is continuous $\forall x \in \mathbb{R}$
        \item $f\brak{x}$ is continuous and differentiable everywhere
        \item $f\brak{x}$ is not diferentiable at two points
        \item $f\brak{x}$ is not differentiable at one point
    \end{enumerate}

    \item 
%6th question
    Let $L=\lim_{x \to0}\frac{a-\sqrt{a^2-x^2-\frac{x^2}{4}}}{x^4}, a>0$. If L is finite, then 

    \hfill(2009)
    
    \begin{multicols}{4}
    \begin{enumerate}
        \item $a=2$ 
        \item $a=1$
        \item $L=\frac{1}{64}$
        \item $L=\frac{1}{32}$
    \end{enumerate}
    \end{multicols}	    


    \item 
%7th question
    Let $f:\mathbb{R} \mapsto \mathbb{R}$ be a function such that $f\brak{x+y}=f\brak{x}+f\brak{y}$, $\forall x,y\in \mathbb{R}$. If $f\brak{x}$ is differentiable at $x=0$, then 

    \hfill(2011)
    
    \begin{enumerate}
	\item $f\brak{x}$ is differentiable only in a finite interval containing zero 
        \item $f\brak{x}$ is continuous $\forall x\in \mathbb{R}$
	\item $f^{\prime}\brak{x}$ is constant $\forall x\in \mathbb{R}$
        \item $f\brak{x}$ is differentiable except at finitely many points 
    \end{enumerate}


    \item 
%8th question
    If $f\brak{x}= 
    \begin{cases}
        -x-\frac{\pi}{2}, & x\leq \frac{\pi}{2} \\
        -\cos x, & \frac{\pi}{2}<x\leq 0 \\
        x-1, & 0<x\leq1 \\
        \ln x, & x>1
    \end{cases}$ 

    \hfill(2011)
    
    \begin{enumerate}
        \item $f\brak{x}$ is continuous at $x=\frac{\pi}{2}$
        \item $f\brak{x}$ is not differentiable at $x=0$
	\item $f^{\prime}\brak{x}$ is differentiable at $x=1$
        \item $f\brak{x}$ is differentiable at $x=\frac{3}{2}$
    \end{enumerate}

    \item 
%9th question
    For every integer $n$, let $a_n$ and $b_n$, be real numbers. Let function $f\brak{x}: \mathbb{IR} \mapsto \mathbb{IR}$ be given by
    $f\brak{x}$= 
    $\begin{cases}
       a_n+\sin\pi x, & \forall x\in \sbrak{2n,2n+1} \\
       b_n+\cos\pi x, & \forall x\in \brak{2n-1,2n}
    \end{cases}$
    for all integers $n$. If $f$ is continuous,then which of the following hold\brak{s} for all $n$ 

    \hfill(2012)
    
    \begin{enumerate}        
        \item $a_{n-1}-b_{n-1}=0$ 
        \item $a_n-b_n=1$ 
        \item $a_n-b_{n+1}=1$ 
        \item $a_{n-1}-b_n=-1$ 
    \end{enumerate}


    \item 
%10th question
	    For $a\in \mathbb{R}$ \brak{\text{the set of all real numbers}}, 
    \begin{align*}
	    a\neq -1, \lim_{n\to\infty}\frac{\brak{1^a+2^a+\cdots+n^a}}{\brak{n+1}^a\sbrak{\brak{na+1}+\brak{na+2}+\cdots+\brak{na+n}}}
	    =\frac{1}{60} \text{ Then } a=
    \end{align*}

    \hfill(JEE Adv.2013)
    
    \begin{multicols}{4}
    \begin{enumerate}
        \item $5$
        \item $7$ 
        \item $\frac{-15}{2}$ 
        \item $\frac{-17}{2}$ 
    \end{enumerate}
    \end{multicols}


    \item 
%11th question
    Let $f: \sbrak{a,b}\mapsto \lsbrak{1},\rbrak{\infty}$ be a continuous function and let $g: \mathbb{R}\mapsto \mathbb{R}$ be defined as 
    $f\brak{x}$= 
    $\begin{cases}
       0, & if x<a, \\
       \int_{a}^{x}{f\brak{t} \, dt}, & if a\leq x\leq b \\
       \int_{a}^{b}{f\brak{t} \, dt}, & if x>b
    \end{cases}$; then 

    \hfill(JEE Adv.2013)
    
    \begin{enumerate}
        \item $g\brak{x}$ is continuous but not differentiable at $a$
        \item $g\brak{x}$ is differentiable on $\mathbb{R}$
        \item $g\brak{x}$ is continuous but not differentiable at $b$
        \item $g\brak{x}$ is continuous and differentiable at either $\brak{a}$ or $\brak{b}$ but not both 
    \end{enumerate}


    \item 
%12th question
	  For every pair of continuous functions $f, g: \sbrak{0, 1}\mapsto \mathbb{R}$ such that max $\cbrak{f\brak{x}: x\in \sbrak{0, 1}}$= max $\cbrak{g\brak{x}: x\in \sbrak{0, 1}}$, the correct statement\brak{s} is\brak{are}: 
    
    \hfill(JEE Adv.2014)
    
    \begin{enumerate}
        \item ${\brak{f\brak{c}}}^2+3f\brak{c}={\brak{g\brak{c}}}^2+3g\brak{c}$ for some $c\in \sbrak{0,1}$
        \item ${\brak{f\brak{c}}}^2+f\brak{c}={\brak{g\brak{c}}}^2+3g\brak{c}$ for some $c\in \sbrak{0,1}$
        \item ${\brak{f\brak{c}}}^2+3f\brak{c}={\brak{g\brak{c}}}^2+g\brak{c}$ for some $c\in \sbrak{0,1}$
        \item ${\brak{f\brak{c}}}^2={\brak{g\brak{c}}}^2$ for some $c\in \sbrak{0,1}$ 
    \end{enumerate}


    \item 
%13th question    
	    Let $g: \mathbb{R}\mapsto \mathbb{R}$ be a differentiable function with $g\brak{0}=0$, $g^{\prime}\brak{0}=0$ and $g^{\prime}\brak{1}\neq 0$. Let $f\brak{x}=
        \begin{cases}
		\frac{x}{\abs{x}}g\brak{x}, & x\neq 0 \\
            0, & x=0
        \end{cases}$ 
	and $h\brak{x}=e^{\abs{x}}$ for all $x\in \mathbb{R}$. Let $\brak{f\circ h}\brak{x}$ denote $f\brak{h\brak{x}}$ and $\brak{h\circ f}\brak{x}$ denote $h\brak{f\brak{x}}$. Then which of the following is\brak{are} true? 

    \hfill(JEE Adv.2015)
    
    \begin{enumerate}        
        \item $f$ is differentiable at $x=0$ 
        \item $h$ is differentiable at $x=0$ 
        \item $f\circ h$ is differentiable at $x=0$ 
        \item $h\circ f$ is differentiable at $x=0$  
    \end{enumerate}


    \item 
%14th question
	    Let $a, b\in \mathbb{R}$ and $f: \mathbb{R}\mapsto \mathbb{R}$ be defined by $f\brak{x}=a\cos \brak{\abs{x^3-x}} +b\abs{x}\sin \brak{\abs{x^3+x}}$. Then $f$ is 
   
    \hfill(JEE Adv.2016)
    
    \begin{enumerate}
        \item differentiable at $x=0$ if $a=0$ and $b=1$
        \item differentiable at $x=1$ if $a=1$ and $b=0$
        \item {NOT} differentiable at $x=0$ if $a=1$ and $b=0$
        \item {NOT} differentiable at $x=1$ if $a=0$ and $b=1$
    \end{enumerate}


    \item 
%15th question
	    Let $f:\sbrak{-\frac{1}{2}, 2}\mapsto \mathbb{R}$ and $g:\sbrak{-\frac{1}{2}, 2}\mapsto \mathbb{R}$ be functions defined by $f\brak{x}=\sbrak{x^2-3}$ and $g\brak{x}=\abs{x}f\brak{x}+\abs{4x-7}f\brak{x}$, where $\sbrak{y}$ denotes the greatest integer less than or equal to $y$ for $y\in \mathbb{R}$. Then 

    \hfill(JEE Adv.2016)
    
    \begin{enumerate}        
        \item $f$ is discontinuous exactly at three points in $\sbrak{-\frac{1}{2}, 2}$
        \item $f$ is discontinuous exactly at four points in $\sbrak{-\frac{1}{2}, 2}$
        \item $g$ is NOT differentiable exactly at four points in $\sbrak{-\frac{1}{2}, 2}$
        \item $g$ is NOT differentiable exactly at five points in $\sbrak{-\frac{1}{2}, 2}$
    \end{enumerate}

%\end{enumerate}

