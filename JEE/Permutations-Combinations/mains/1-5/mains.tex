\iffalse
\title{Assignment}
\author{Banapuram Pranay Kumar}
\section{mains}
\fi



\item Total number of four digit odd numbers that can be formed using $0,1,2,3,5,7$(using repetition allowed) are 
\hfill(2002)
\begin{enumerate}

    \item $216$
    
    \item $400$
    \item $720$
    \item $375$\\
\end{enumerate}

\item Number greater than $1000$ but less than $4000$ is  formed using the digits $0,1,2,3,4$(repetition allowed).Their number is
\hfill(2002)\\
\begin{enumerate}
    \item $125$
    \item $105$
    \item $375$
    \item $625$
\end{enumerate}

\item Five digit numbers divisible by $3$ is formed using $0,1,2,3,4$ and $5$ without repetition.Total number of such numbers are
\hfill(2002)
\begin{enumerate}
    \item $312$
    \item $3125$
    \item $120$
    \item $216$ \\
\end{enumerate}
\item The sum of integers from $1$ to $100$ that are divisible by $2$ or $5$ is
\hfill \brak{2002}
\begin{enumerate}
    \item $30000$
    \item $3050$
    \item $3600$
    \item $3250$ 
\end{enumerate}

\item If $^nC_r$ denotes the number of combinations of $n$ things taken $r$ at a time, then the expression $^nC_{r+1} + ^nC_{r-1} = 2  ^nC_r$ equals
\hfill(2003)
\begin{enumerate}
    \item$^{n+1}C_{r+1}$
    \item$^{n+2}C_r$
    \item$^{n+2}C_{r+1}$
    \item$^nC_r$\\
\end{enumerate}



