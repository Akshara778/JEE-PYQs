\iffalse
\title{Chapter 4 Permutations and Combinations}
\author{EE24BTECH11012 - Bhavanisankar G S}
\section{mains}
\fi
%\begin{enumerate}
	\item A student is to answer $10$ out of $13$ questions in an examination such that he must choose atleast four from the first five questions. The number of choices available to him is  \hfill \brak{2003}	
       \begin{multicols}{4}		
\begin{enumerate} 
    \item  $346$
    \item  $140 $
    \item  $196$
    \item  $280$
      \end{enumerate}
      \end{multicols}
\item The number of ways in which $6$ men and $5$ women can dine around a round table if no two women are to sit together is given by \hfill \brak{2003}
	\begin{multicols}{4}	
 \begin{enumerate}
 \item $7!\times5!$
 \item $6!\times 5!$
 \item $30!$
 \item $5!\times4!$
 \end{enumerate}
	\end{multicols}
\item How many ways are there to arrange the letters in the word \emph{GARDEN} with vowels iin alphabetical order \hfill \brak{2004}
	\begin{multicols}{4}
		\begin{enumerate}
     \item $480$
     \item $240$
     \item $360$
     \item $120$
 \end{enumerate}
	\end{multicols}
\item The number of ways of distributing 8 identical balls in $3$ distinct boxes so that none of the boxes is empty is \hfill \brak{2004}
	\begin{multicols}{4}
		\begin{enumerate}
			\item $ \comb{8}{3}$
     \item $21$
     \item $3^8$
     \item $5$
 \end{enumerate}
	\end{multicols}
\item If the letters of the word \emph{SACHIN} are arranged in all the possible ways and these words are written out as in dictionary, then the word \emph{SACHIN} appears at the serial number \hfill \brak{2005}
	\begin{multicols}{4}
		\begin{enumerate}
     \item $601$
     \item $600$
     \item $603$
     \item $602$
 \end{enumerate}
	\end{multicols}
\item At an election, a voter may vote for any number of candidates, not greater than the number to be elected. There are $10$ candidates and $4$ are of be selected, if a voter votes for atleast one candidate, then the number of ways in which he can vote is \hfill\brak{2006}
	\begin{multicols}{4}
        \begin{enumerate}
 \item $5040$
 \item $6210$
 \item $385$
 \item $1110$
        \end{enumerate}
	\end{multicols}
\item The set $S={1,2,3,...,12}$ is to be partitioned into three sets A,B and C of equal size. Thus, $A \cup B \cup C = S, A \cap B = B \cap C = A \cap C = \phi $. The number of ways to partition $S$ is \hfill\brak{2007}
	\begin{multicols}{2}
	\begin{enumerate}
     \item $ \frac{12!}{(4!)^3}$
     \item $\frac{12!}{(4!)^4}$
     \item $\frac{12!}{3!(4!)^3}$
     \item $\frac{12!}{3!(4!)^4}$
        \end{enumerate}
	\end{multicols}
\item In a shop, there are five types of ice-creams available. A child buys six ice-creams.\\\textbf{Statement1}: The number of different ways in which the child can buy six ice-creams is $ \comb{10}{5} $.\\\textbf{Statement2}:The number of different ways in which the child can buy six ice-creams is equal to the number of different ways of arranging 6 A's and 4 B's in a row. \hfill\brak{2008}
 \begin{enumerate}
     \item Statement 1 is false, Statement 2 is true.
     \item Statement 1 is true, Statement 2 is true, Statement 2 is the correct explanation of Statement 1.
     \item Statement 1 is true, Statement 2 is true, Statement 2 is not a correct explanation of statement 1.
     \item Statement 1 is true, Statement 2 is false.
 \end{enumerate}
\item  How many different words can be formed by jumbling the letters in the word \emph{MISSISSIPPI} in which no two S are adjacent ?
	\begin{multicols}{2}
	\begin{enumerate}
		\item $8 \comb{6}{4} \comb {7}{4}$
		\item $6\times7 \comb{8}{4}$
		\item $6\times8$$\comb{7}{4}$
		\item $ 7 \comb{6}{4} \comb {8}{4}$
        \end{enumerate}
	\end{multicols}
\item From $6$ different novels and $3$ different dictionaries, $4$ novels and $1$ dictionary are to be selected and arranged in a row on a shelf so that the dictionary is always in the middle. Then the number of such arrangements is: \hfill\brak{2009}
 \begin{enumerate}
     \item $at least 500 but less than 750$
     \item $at least 750 but less than 1000$
     \item $at least 1000$
     \item $less than 500$
 \end{enumerate}
\item There are two urns. Urn A has $3$ distinct red balls while Urn B has $9$ distinct blue balls. From each urn, two balls are taken at random and then transferred to the other. The number of ways in which this can be done is: \hfill\brak{2010}
	\begin{multicols}{4}
	\begin{enumerate}
     \item $36$
     \item $66$
     \item $108$
     \item $3$
        \end{enumerate}
	\end{multicols}
\item \textbf{Statement 1}: The number of ways of distributing $10$ identical balls in $4$ identical boxes such that no box is empty is $\comb{9}{3}$.\\\textbf{Statement 2}: The number of ways of choosing any three $3$ places from $9$ different places is $\comb{9}{3}$. \hfill\brak{2011}
 \begin{enumerate}
     \item Statement 1 is true, statement 2 is true; Statement 2 is the correct explanation of Statement 1.
     \item Statement 1 is true, Statement 2 is true; Statement 2 is not the correct explanation of Statement 1.
     \item Statement 1 is true, Statement 2 is false.
     \item Statement 1 is false, Statement 2 is true.
 \end{enumerate}
\item These are $10$ points in a plane, out of which 6 are collinear. If \emph{N} is the number of triangles formed by these points, then; \hfill\brak{2012}
	\begin{multicols}{2}
	\begin{enumerate}
		\item $ \emph{N} \le100$
		\item $100< \emph{N} \le140$
		\item $140< \emph{N} \le190$
		\item $ \emph{N} >190$
        \end{enumerate}
	\end{multicols}
\item Assuming the balls to be identical except for the difference in colours, The number of ways in which one or more balls can be selected from 10 white, 9 green and 7 black balls. \hfill \brak{2012}
	\begin{multicols}{4}
	\begin{enumerate}
     \item $880$
     \item $629$
     \item $630$
     \item $879$
        \end{enumerate}
	\end{multicols}
\item Let $T_n$ be the set of all possible triangles formed by joining the vertices of a n-sided regular polygon. If $T_{n+1} - T_n = 10$, then the value of n is : \hfill \brak{JEE MAIN 2013}
	\begin{multicols}{4}
	\begin{enumerate}
     \item $7$
     \item $5$
     \item $10$
     \item $8$
        \end{enumerate}
	\end{multicols}
%\end{enumerate}
