\iffalse

\title{Assignment}
\author{Arjun Pavanje}
\section{mcq-multiple}
\fi
\item The principal value of $\sin ^{-1}\brak{\sin \brak{\frac{2\pi}{3}}}$ is
\hfill \brak{1986-2Marks}
\begin{enumerate}
\begin{multicols}{2}
\item $-\frac{2\pi}{3}$ 
\columnbreak
\item $\frac{2\pi}{3}$ 
\end{multicols}
\begin{multicols}{2}
\item $\frac{4\pi}{3}$ 
\columnbreak
\item none
\end{multicols}
\end{enumerate}
\item If $\alpha=3\sin ^{-1}\brak{\frac{6}{11}}$ and $\beta=3\cos ^{-1}\brak{\frac{4}{9}}$, where the inverse trigonometric functions take only the principal values, then the correct option\brak{\text{s}} is\brak{\text{are}}
\hfill \brak{JEE Adv.2015}
\begin{enumerate}
\begin{multicols}{2}
\item $\cos \brak{\beta}>0$ 
\columnbreak
\item $\sin \brak{\beta}<0$
\end{multicols}
\begin{multicols}{2}
\item $\cos \brak{\alpha + \beta} > 0$ 
\columnbreak
\item $\cos \brak{\alpha}<0$
\end{multicols}
\end{enumerate}
\item For non-negative integers $n$, let 
\begin{align*}
f\brak{n}= \frac{\sum_{k=0}^{n} \sin \brak{\frac{k+1}{n+2}\pi}\sin \brak{\frac{k+2}{n+2}\pi}}{\sum_{k=0}^{n} \sin ^2 \brak{\frac{k+1}{n+2}\pi}}
\end{align*}
Assuming $\cos ^{-1}\brak{x}$ takes values in $\sbrak{0,\pi}$, which of the following options is/are correct
\hfill \brak{JEE Adv.2019}
\begin{enumerate}
\item $\lim_{n \rightarrow \infty)} f\brak{n} = \frac{1}{2}$ 
\item $f\brak{4}=\frac{\sqrt{3}}{2}$
\item If $\alpha = \tan \brak{\cos ^{-1}\brak{f\brak{6}}}$, then $\alpha ^2 + 2\alpha -1 =0$
\item $\sin \brak{7\cos ^{-1}\brak{f\brak{5}}}=0$
\end{enumerate}
