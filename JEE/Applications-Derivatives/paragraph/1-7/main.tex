\iffalse
\title{Assignment}
\author{Y.Harsha Vardhan Reddy}
\section{paragraph}
\fi
\item {
If a continuous function $f$ defined on the real line $R$, assumes positive and negative values in $R$ then the equation $f\brak{x}=0$ has a root in $R$. For example, if it is known that a continuos function $f$ on $R$ is positive at some point and its minimum value is negative then the eequation $f\brak{x}=0$ has a root in $R$.\\
 Consider $f\brak{x}=ke^x-x$ for all real $x$ where $k$ is a real constant.
 % \begin{enumerate}
 \item The line $y=x$ meets curve $y=ke^x$ for $k \le 0$ at
 \hfill {\brak{2007-4marks}}\\
    \begin{enumerate}
\begin{multicols}{2}
\item no point
\columnbreak
\item one point
\end{multicols}
\begin{multicols}{2}
\item two points
\item more than two points
\end{multicols}
\end{enumerate}

  



\item  The positive value of $k$ for which $ke^x-x=0$ has only one root is
\hfill {\brak{2007-4marks}}\\
\begin{enumerate}
\begin{multicols}{2}
\item $\frac{1}{e}$
\columnbreak
\item 1
\end{multicols}
\begin{multicols}{2}
\item $e$
\item $\log_e{2}$
\end{multicols}
\end{enumerate}




\item For $k>0$, the set of all values of $k$ for which equation $ke^x-x=0$ has two distinct roots is 
\hfill {\brak{2007-4marks}}\\
\begin{enumerate}
    \begin{multicols}{2}
    \item $\brak{0,\frac{1}{e}}$
    \columnbreak
    \item $\brak{\frac{1}{e},1}$
    \end{multicols}
    \begin{multicols}{2}
    \item $\brak{\frac{1}{e},\infty}$
    \item $\brak{0,1}$
    \end{multicols}
\end{enumerate}
}
\item {
Let \begin{align}f\brak{x}=\brak{1-x}^2 \sin^2 x + x^2 for all x \in \mathbb{IR} \end{align} and let \begin{align}g\brak{x}=
\int_{1}^{x} \brak{\frac{2(t-1)}{t+1} - \ln t}  f(t) \, dt for all x \in \brak{1 ,\infty}.\end{align}
\item Consider the statements:\\
P : There exists some $x \in \mathbb{R}$ such that $f\brak{x} + 2x = 2\brak{1+x^2}$\\
Q : There exists some $x \in\mathbb{R}$ such that $2f\brak{x} + 1 = 2x\brak{1+x}$\\
    Then
    \hfill{\brak{2012}}
\begin{enumerate}
\item both $P$ and $Q$ are true
\item $P$ is true and $Q$ is false
\item $P$ is false and $Q$ is true
\item both $P$ and $Q$ are false
\end{enumerate}



\item Which of the following is true?
\hfill{\brak{2012}}
\begin{enumerate}

\item $g$ is increasing on $\brak{1,\infty}$
\item $g$ is decreasing on $\brak{1,\infty}$
\item $g$ is increasing in (1,2) and decreasing on $\brak{2,\infty}$
\item $g$ is decreasing in (1,2) and increasing on $\brak{2,\infty}$

\end{enumerate}

}
\item {
Let $f\brak{x} : \sbrak{0,1} \to\mathbb{R}$
(the set of all real numbers) be a function. Suppose the function $f$ is twice differentiable , $f\brak{0}=f\brak{1}=0$ and satisfies \begin{align} f''\brak{x}-2f'\brak{x}+f\brak{x} \geq e^x , x \in [0,1].\end{align} 

\item Which of the following is true for interval     $$0<x<1$$?
\hfill{\brak{JEE Adv. 2013}}

\begin{enumerate}
\begin{multicols}{2}
\item $0<f\brak{x}< \infty$ 
\columnbreak
\item $ -\frac{1}{2} <f\brak{x}< \frac{1}{2}$
\end{multicols}
\begin{multicols}{2}
\item $-\frac{1}{4}<f\brak{x}<1$
\item $-\infty <f\brak{x}<0$
\end{multicols}
\end{enumerate}


\item If the function $e^{-x}f(x)$ assumes its minimum in the interval \sbrak{0,1} at $x=\frac{1}{4}$, which of the following is true?


\hfill{\brak{JEE Adv. 2013}}

\begin{enumerate}

\item $f'\brak{x}<f\brak{x}$ , $\frac{1}{4}<x<\frac{3}{4}$ \\

\item $f'\brak{x}>f\brak{x}$ ,$0<x<\frac{1}{4}$ \\ 

\item $f'\brak{x}<f\brak{x}$ ,$0<x<\frac{1}{4}$ \\

\item $f'\brak{x}<f\brak{x}$ ,$\frac{3}{4}<x<1$ \\


\end{enumerate}
}
% \end{enumerate}
