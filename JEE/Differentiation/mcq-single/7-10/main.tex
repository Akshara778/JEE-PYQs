\iffalse
  \title{Assignment}
  \author{AI24BTECH11022 - Pabbuleti Venkata Charan Teja}
  \section{mcq-single}
\fi

%\begin{enumerate}[start=7]
\item
If $f\brak{x}$ is a twice differentiable function and given that $f\brak{1}=1,f\brak{2}=4,f\brak{3}=9$, then 

\hfill{(2005S)}

\begin{enumerate}
\item
$f^{\prime\prime}\brak{x}=2$ $,\forall$ $x\in\brak{1,3}$

\item $f^{\prime\prime}\brak{x}=f^{\prime}\brak{x}=5$ for some $x\in\brak{2,3}$

\item $f^{\prime\prime}\brak{x}=3$ $,\forall$ $x\in\brak{2,3}$

\item $f^{\prime\prime}\brak{x}=2$ for some $x\in\brak{1,3}$
\end{enumerate}

\item
$\frac{d^{2}x}{dy^{2}}$ \hfill{(2007-3 marks)}

\begin{multicols}{2}
\begin{enumerate}
\item$\brak{\frac{d^{2}y}{dx^{2}}}^{-1}$
\item$-\brak{\frac{d^{2}y}{dx^{2}}}^{-1}\brak{\frac{dy}{dx}}^{-3}$
\item$\brak{\frac{d^{2}y}{dx^{2}}}\brak{\frac{dy}{dx}}^{-2}$
\item$-\brak{\frac{d^{2}y}{dx^{2}}}\brak{\frac{dy}{dx}}^{-3}$
\end{enumerate}    
\end{multicols}

\item
Let $g\brak{x}=\text{log}f\brak{x}$ is twice differentiable positive function on $\brak{0,\infty}$ such that ${f\brak{x+1}=xf\brak{x}}$. Then, for $N=1,2,3,\dots$ 

\hfill{(2008)}

$g^{\prime\prime}\brak{N+\frac{1}{2}}-g^{\prime\prime}\brak{\frac{1}{2}}=$

\begin{enumerate}
\item$-4 \cbrak{1+\frac{1}{9}+\frac{1}{25}+\dots+\frac{1}{\brak{2N-1}^{2}}}$

\item $4\cbrak{1+\frac{1}{9}+\frac{1}{25}+\dots+\frac{1}{\brak{2N-1}^{2}}}$

\item $-4\cbrak{1+\frac{1}{9}+\frac{1}{25}+\dots+\frac{1}{\brak{2N+1}^{2}}}$

\item $4\cbrak{1+\frac{1}{9}+\frac{1}{25}+\dots+\frac{1}{\brak{2N+1}^{2}}}$
\end{enumerate}

\item
Let $f\,:\,\sbrak{0,2}\to\mathbb{R}$ be a function which is continuous on $\sbrak{0,2}$ and is differentiable on $\brak{0,2}$ with $f\brak{0}=1$. Let $F\brak{x}=\int \limits_{0}^{x^{2}}f\brak{\sqrt{t}}dt$ for $x\in\sbrak{0,2}$. If $F^{\prime}\brak{x}=f^{\prime}\brak{x}$ for all ${x\in\brak{0,2}}$ then $F\brak{2}$ equals \hfill{(JEE Adv. 2014)}

\begin{multicols}{2}
\begin{enumerate}
\item$e^{2}-1$
\item$e^{4}-1$
\item$e-1$
\item$e^{4}$
\end{enumerate} 
\end{multicols}
%\end{enumerate}