\iffalse
  \title{Assignment}
  \author{AI24BTECH11022 - Pabbuleti Venkata Charan Teja}
  \section{mcq-multiple}
\fi

%\begin{enumerate}

\item
Let $f\,:\,\mathbb{R}\to\mathbb{R}$, $g\,:\,\mathbb{R}\to\mathbb{R}$ and $h\,:\,\mathbb{R}\to\mathbb{R}$ be differentiable functions such that ${f\brak{x}=x^{3}+3x+2}$, $g\brak{f\brak{x}}=x$ and $h\brak{g\brak{g\brak{x}}}=x$, $\forall x\in\mathbb{R}$. Then 

\hfill{(JEE Adv. 2016)}

\begin{multicols}{2}
\begin{enumerate}

\item$g^{\prime}\brak{2}=\frac{1}{15}$
\item$h^{\prime}\brak{1}=666$
\item$h\brak{0}=16$
\item$h\brak{g\brak{3}}=36$ 

\end{enumerate}
\end{multicols}

\item 
For every twice differential function ${f\,:\,\mathbb{R}\to\sbrak{-2,2}}$ with $\brak{f\brak{0}}^{2}+\brak{f^{\prime}\brak{0}}^{2}=85$, which of the following statement(s) is(are) TRUE?

\hfill{(JEE Adv. 2018)}

\begin{enumerate}
\item There exist $r,s\in\mathbb{R}$, where $r<s$, such that $f$ is one-one on the open interval $\brak{r,s}$

\item There  exists $x_{0}\in\brak{-4,0}$ such that $\abs{f^{\prime}\brak{x_{0}}}\leq1$

\item $\lim \limits_{x\to\infty}f\brak{x}=1$

\item There exists $\alpha\in\brak{-4,4}$ such that ${f\brak{\alpha}+f^{\prime\prime}\brak{\alpha}=0}$ and ${f^{\prime}\brak{\alpha}\not=0}$
\end{enumerate}

\item 
For any positive integer $n$, define ${f_{n}\brak{x}=\sum\limits_{j=1}^{n}\tan^{-1}\brak{\frac{1}{1+\brak{x+j}\brak{x+j-1}}},\forall\,x\in\brak{0,\infty}}$

Here, the inverse trigonometric function $\tan^{-1}x$ assumes values in $\brak{-\frac{\pi}{2},\frac{\pi}{2}}$.

Then, which of the following statement(s) is (are) TRUE? \hfill{(JEE Adv. 2018)}

\begin{enumerate}
\item $\sum\limits_{j=1}^{5}\tan^{2}\brak{f_{j}\brak{0}}=55$

\item $\sum\limits_{j=1}^{10}\brak{1+f_{j}^{\prime}\brak{0}}\brak{\sec^{2}\brak{f_{j}\brak{0}}}=10$

\item For any fixed positive integer $n$, $\lim\limits_{x\to\infty}\tan\brak{f_{n}\brak{x}}=\frac{1}{n}$

\item For any fixed positive integer $n$, $\lim\limits_{x\to\infty}\sec^{2}\brak{f_{n}\brak{x}}=1$
\end{enumerate}

\item 
Let $f\,:\,\brak{0,\pi}\to\mathbb{R}$ be a twice differentiable function such that ${\lim\limits_{t\to x}\frac{f\brak{x}\sin t-f\brak{t}\sin x}{t-x}=\sin^{2}x,\,\forall x\in\brak{0,\pi}}$. If $f\brak{\frac{\pi}{6}}=-\frac{\pi}{12}$, then which of the following statement(s) is(are) TRUE? 

\hfill{(JEE Adv. 2018)}

\begin{enumerate}
\item $f\brak{\frac{\pi}{4}}=\frac{\pi}{4\sqrt{2}}$

\item $f\brak{x}<\frac{x^{4}}{6}-x^{2},\,\forall x\in\brak{0,\pi}$

\item There exists $\alpha\in\brak{0,\pi}$ such that $f^{\prime}\brak{\alpha}=0$

\item $f^{\prime\prime}\brak{\frac{\pi}{2}}+f\brak{\frac{\pi}{2}}=0$
\end{enumerate}
%\end{enumerate}