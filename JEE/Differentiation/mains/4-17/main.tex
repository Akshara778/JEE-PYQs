
\iffalse
  \title{Differentiation}
  \author{Suraj Kolluru}
  \section{mains}
\fi

%   \begin{enumerate}
\item Let $f\brak{x}$ be a polynomial function of second degree.If $f\brak{1}$=$f\brak{-1}$ and $a,b,c$ are in A.P, then $f^{\prime} \brak{a},f^{\prime}\brak{b},f^{\prime}\brak{c}$ are in
    \hfill[2003]
    \begin{enumerate}
    \item Arthmetic-Geometric Progression
    \item A.P
    \item G.P
    \item H.P
    
    \end{enumerate}
    \item If x=$e^{y+e^{y+e^{y+---\infty}}}$ ,$x>0$, then $\frac{dy}{dx}$ is
    \hfill[2004]
    
    \begin{multicols}{4}
    \begin{enumerate}
    \item $\frac{1+x}{x}$
    \item $\frac{1}{x}$
    \item $\frac{1-x}{x}$
    \item $\frac{x}{1+x}$
\end{enumerate}
\end{multicols}
\item The value of $a$ for which the sum of the squares of the roots of the equation $x^2-\brak{a-2}x-a-1=0$ assume the least value is
\hfill[2005]
\begin{multicols}{4}
\begin{enumerate}
    \item 1
    \item 0
    \item 3
    \item 2
    \end{enumerate}
    \end{multicols}
    \item If the roots of the equation $x^2-bx+c=0$ be   two  consecutive integers, then $b^2-4c$  equals  to
    \hfill[2005]
    \begin{multicols}{4}
    \begin{enumerate}
        \item -2
        \item 3
        \item 2
        \item 1
    \end{enumerate}
    \end{multicols}
    \item Let $f:R\rightarrow R$ be a differentiable function having $f\brak{2}=6$,$f^{\prime}\brak{2}=\brak{\frac{1}{48}}$.Then $\lim_{x\to\ 2} \int_{6}^{f(x)} \frac{4t^3}{x-2} dt$ equals to
    \hfill[2005]
    \begin{multicols}{4}
        
   
    \begin{enumerate}
    \item 24
    \item 36
    \item 12
    \item 18
    \end{enumerate}
     \end{multicols}
    \item The set of points where $f(x)=\frac{x}{1+\abs{x}}$ is differentiable is \hfill[2006]
    \begin{multicols}{2}
    
    \begin{enumerate}
    \item $\brak{-\infty,0}\bigcup\brak{0,\infty}$
    \item $\brak{-\infty,-1}\bigcup\brak{-1,\infty}$
    \item $\brak{-\infty,\infty}$
    \item $\brak{0,\infty}$
    \end{enumerate}
    \end{multicols}
    \item If $x^m\cdot y^n={x+y}^{m+n}$ ,then $\frac{dy}{dx}$ is \hfill[2006]
    \begin{multicols}{4}
        
  
    \begin{enumerate}
    \item $\frac{y}{x}$
    \item $\frac{x+y}{xy}$
    \item $xy$
    \item $\frac{x}{y}$
    \end{enumerate}
    \end{multicols}
    \item Let $y$ be an implicit function of $x$ defined by $x^{2x}-2x^x\cot{y}-1=0$.Then $y^{\prime}\brak{1}$ equals
    \hfill[2009]
    \begin{multicols}{4}
    \begin{enumerate}
    \item 1
    \item $\log2$
    \item -$\log2$
    \item -1
    \end{enumerate}
    \end{multicols}
    \item Let $f:\brak{-1,1}\rightarrow R$ be a differentiable function with $f\brak{0}=-1$ and $f^{\prime}\brak{0}=1$.Let $g\brak{x}=\brak{f\brak{2f\brak{x}+2}}^2$.Then $g^{\prime}\brak{0}=$
	    \hfill[2010]
     \begin{multicols}{4}
         
     
	    \begin{enumerate}
	\item -4
	\item 0
	\item -2
	\item 4
	    \end{enumerate}
     \end{multicols}
    \item $\frac{d^2x}{dy^2}$ equals:\hfill[JEE M 2013]
    \begin{multicols}{2}
        
    
	    \begin{enumerate}
		    \item  $-\brak{{\frac{d^2y}{dx^2}}}^{-1}\brak{{\frac{dy}{dx}}}^{-3}$
		    \item $\brak{{\frac{d^2y}{dx^2}}}\brak{{\frac{dy}{dx}}}^{-2}$

		    \item $-\brak{{\frac{d^2y}{dx^2}}}\brak{{\frac{dy}{dx}}}^{-3}$
		    \item $\brak{\frac{d^2y}{dx^2}}^{-1}$
	    \end{enumerate}
     \end{multicols}
    \item If $y=\sec\brak{\tan^{-1}x}$,then $\frac{dy}{dx}$ at $x$=1 is equal to:
    \hfill[JEE M 2013]

\begin{multicols}{2}
    

     \begin{enumerate}
     \item $\frac{1}{\sqrt{2}}$
     \item $\frac{1}{2}$
     \item 1
     \item $\sqrt{2}$
     \end{enumerate}
     \end{multicols}
      \item If $g$ is the inverse of a function $f$ and $f^{\prime}\brak{x}$=$\frac{1}{1+x^5}$,then $g^{\prime}\brak{x}$ is equal to:\\    
      \hfill[$JEE$ M 2014]
      \begin{multicols}{2}
          
      
      \begin{enumerate}
      \item $\frac{1}{1+\brak{{g(x)}}^5}$
      \item 1+$\brak{{g(x)}}$
      \item $1+x^5$
      \item $5x^4$
      \end{enumerate}
      \end{multicols}
      \item If $x$=-1 and $x$=2 are extreme points of $f\brak{x}=\alpha\log\abs{x}+\beta x^2+x$ then 
	      \hfill[JEE M 2014]
       \begin{multicols}{2}
           
       
	      \begin{enumerate}
		      \item $\alpha=2,\beta=-\frac{1}{2}$
		      \item $\alpha=2,\beta=\frac{1}{2}$
		      \item $\alpha=-6,\beta=\frac{1}{2}$
		      \item $\alpha=-6,\beta=-\frac{1}{2}$
	      \end{enumerate}
       \end{multicols}
      \item If for $x\in\brak{0,\frac{1}{4}}$,the derivative of $\tan^{-1}\brak{\frac{6x\sqrt{2}}{1-9x^3}}$ is $\sqrt{x}\cdot g\brak{x}$,then $g\brak{x}$ equals:
	      \hfill[JEE M 2017]
       \begin{multicols}{2}
           
    
	      \begin{enumerate}
		      \item $\frac{3}{1+9x^3}$
		      \item $\frac{9}{1+9x^3}$
		      \item $\frac{3x\sqrt{x}}{1-9x^3}$
		      \item $\frac{3x}{1-9x^3}$
        
	      \end{enumerate}
       \end{multicols}

    
% \end{enumerate}
