\iffalse
  \title{Trignometric Functions and Equations}
  \author{EE24BTECH11002 - AGAMJOT SINGH}
  \section{mcq-single}
\fi
    %Question5
    \item The general solution of the trigonometric equation $\sin {x} + \cos{x} = 1$ is given by:
        \hfill{\brak{1981 - 2 Marks}}
        \begin{enumerate}
            \item $x = 2n\pi;\;n = 0,\pm1,\pm2\;\dots$
            \item $x = 2n\pi+\frac{\pi}{2};\;n = 0,\pm1,\pm2\;\dots$
            \item $x = n\pi+\brak{-1}^n\frac{\pi}{4}-\frac{\pi}{4};\;n = 0,\pm1,\pm2\;\dots$
            \item none of these
        \end{enumerate}

    %Question6
    \item The value of the expression $\sqrt{3}\;\cosec{20\degree}-\sec{20\degree}$ is equal to
        \hfill{\brak{1988 - 2 Marks}}
        \begin{enumerate}
                \item $2$
                \item $2\frac{\sin{20\degree}}{\sin{40\degree}}$
                \item $4$
                \item $4\frac{\sin{20\degree}}{\sin{40\degree}}$
        \end{enumerate}

    %Question7
    \item The general solution of 
	\begin{multline*}
		    \sin{x}-3\sin{2x} + \sin{3x} = \\\cos{x}-3\cos{2x} + \cos{3x}
	\end{multline*}
        
        \hfill{\brak{1989 - 2 Marks}}
        \begin{enumerate}
                \item $n\pi+\frac{\pi}{8}$
                \item $\frac{n\pi}{2}+\frac{\pi}{8}$
                \item $\brak{-1}^n\frac{n\pi}{2}+\frac{\pi}{8}$
                \item $2n\pi+\cos^{-1}{\frac{3}{2}}$
        \end{enumerate}


    %Question8
    \item The equation $\brak{\cos{p}-1}x^2+\brak{\cos{p}}x+\sin{p}=0$ in the variable $x$, has real roots. Then $p$ can take any value in the interval
        
        \hfill{\brak{1990 - 2 Marks}}
        \begin{enumerate}
                \item $\brak{0,2\pi}$
                \item $\brak{-\pi,0}$
                \item $\brak{-\frac{\pi}{2},\frac{\pi}{2}}$  
                \item $\brak{0,\pi}$
        \end{enumerate}

    %Question9
    \item Number of solutions of the equation $\tan{x}+\sec{x} = 2\cos{x}$ lying in the interval $\brak{0, 2\pi}$ is
        
        \hfill{\brak{1993 - 1 Marks}}
        \begin{enumerate}
                \item $0$
                \item $1$
                \item $2$
                \item $3$
        \end{enumerate}

    %Question10
    \item Let $0<x<\frac{\pi}{4}$ then $\brak{\sec{2x} - \tan{2x}}$ equals
        
        \hfill{\brak{1994}}
        \begin{enumerate}
                \item $\tan{\brak{x-\frac{\pi}{4}}}$
                \item $\tan{\brak{\frac{\pi}{4}-x}}$
                \item $\tan{\brak{x+\frac{\pi}{4}}}$ 
                \item $\tan^{2}{\brak{x+\frac{\pi}{4}}}$
        \end{enumerate}

    %Question11
    \item Let n be a positive integer such that $\sin{\frac{\pi}{2n}} + \cos{\frac{\pi}{2n}} = \frac{\sqrt{n}}{2}$. Then
        
        \hfill{\brak{1994}}
        \begin{enumerate}
                \item $6\le n\le8$
                \item $4<n\le8$
                \item $4\le n\le8$  
                \item $4<n<8$
        \end{enumerate}

    %Question12
    \item If $\omega$ is an imaginary cube root of unity then the value of $\sin{\brak{\brak{\omega^{10} + \omega^{23}}\pi - \frac{\pi}{4}}}$ is
    
        \hfill{\brak{1994}}
        \begin{enumerate}
                \item $-\frac{\sqrt{3}}{2}$
                \item $-\frac{1}{\sqrt{2}}$
                \item $-\frac{1}{\sqrt{2}}$
                \item $\frac{\sqrt{3}}{2}$
        \end{enumerate}

    %Question13
	\item \begin{multline*}
		3\brak{\sin{x} - \cos{x}}^4 + 6\brak{\sin{x} + \cos{x}}^4 + \\ 4\brak{\sin^6{x}+\cos^6{x}} =
	\end{multline*}
        
        \hfill{\brak{1995S}}
        \begin{enumerate}
                \item $11$
                \item $12$
                \item $13$
                \item $14$
        \end{enumerate}   

    %Question14
    \item The general values of $\theta$ satisfying the equation $2\sin^2{\theta}-3\sin{\theta}-2=0$ is
        
        \hfill{\brak{1995S}}
        \begin{enumerate}
                \item $n\pi + \brak{-1}^n\frac{\pi}{6}$
                \item $n\pi + \brak{-1}^n\frac{\pi}{2}$
                \item $n\pi + \brak{-1}^n\frac{5\pi}{6}$ 
                \item $n\pi + \brak{-1}^n\frac{7\pi}{6}$
        \end{enumerate}

    %Question15
    \item $\sec^2{\theta} = \frac{4xy}{\brak{x+y}^2}$ is true if and only if
        
        \hfill{\brak{1996 - 1 Mark}}
        \begin{enumerate}
                \item $x+y=0$
                \item $x=y,x\neq0$
                \item $x=y$ 
                \item $x\neq0,y\neq0$
        \end{enumerate}
        
    %Question16
\item In a triangle $PQR$, $\angle R = \frac{\pi}{2}$. If $\tan{\frac{P}{2}}$ and $\tan{\frac{Q}{2}}$ are the roots of the equation $ax^2+bx+c=0 \;\brak{a\neq0}$ then
        
        \hfill{\brak{1999 - 2 Marks}}
        \begin{enumerate}
                \item $a+b=c$
                \item $b+c=a$
                \item $a+c=b$ 
                \item $b=c$
        \end{enumerate}

    %Question17
    \item Let $f\brak{\theta} = \sin{\theta}\brak{\sin{\theta} + \sin{3\theta}}$. Then $f\brak{\theta}$ is
        
        \hfill{\brak{2000S}}
        \begin{enumerate}
                \item $\ge0$ only when $\theta\newline\ge0$
                \item $\le0$ for all real $\theta$
                \item $\ge0$ for all real $\theta$
                \item $\le0$ only when $\theta\le0$
        \end{enumerate}

    %Question18
    \item The number of distinct real roots of
    \begin{align*}
	    \mydet{
		\sin{x}&\cos{x}&\cos{x}\\
    		\cos{x}&\sin{x}&\cos{x}\\
		\cos{x}&\cos{x}&\sin{x}}
    \end{align*}

        \hfill{\brak{2001S}}
        \begin{enumerate}
                \item $0$
                \item $2$
                \item $1$
                \item $3$
        \end{enumerate}

    %Question19
    \item The maximum value of $\brak{\cos{\alpha_1}}\brak{\cos{\alpha_2}}\brak{\cos{\alpha_3}}\dots\brak{\cos{\alpha_n}}$ under the restrictions
   	\begin{align*} 
		0\le\alpha_1,\alpha_2,\dots\alpha_n\le\frac{\pi}{2}
	\end {align*} and 
	\begin{align*}
		\brak{\cot{\alpha_1}}\brak{\cot{\alpha_2}}\brak{\cot{\alpha_3}}\dots\brak{\cot{\alpha_n}} = 1
	\end{align*}
        \hfill{\brak{2001S}}
        \begin{enumerate}
                \item $\frac{1}{2^{\frac{n}{2}}}$
                \item $\frac{1}{2^{n}}$
                \item $\frac{1}{2n}$
                \item $1$
        \end{enumerate}
