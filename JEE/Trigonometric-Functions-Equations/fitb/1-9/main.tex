\iffalse
\title{Trignometric Functions and Equations}
\author{EE24BTECH11001- ADITYA TRIPATHY}
\section{fitb}
\fi

    \item Suppose $\sin^3{x}\sin3x = \sum_{m=0}^{n} C_m \cos x $ is an identity in $x$, where $C_0, C_1, \cdots , C_n$ are constants and $C_n \neq 0$ then the value of n is
        \hfill{\brak{1981 - 2 Marks}}


    \item The solution set of the system of equations $x + y = \frac{2\pi}{3}$, $\cos x + \cos y = \frac{3}{2}$, where $x$ and $y$ are real,is 
        \hfill{\brak{1987 - 2 Marks}}
    \item The set of all $x$ in the interval $\sbrak{0,\pi}$ for which $2 \sin^2 x -3\sin x +1 \ge 0$, is 

        \hfill{\brak{1987 - 2 Marks}}



    \item The sides of a triangle in a given circle subtend angles $\alpha$, $\beta$, $\gamma$. The minimum value of arithmetic mean of $\cos \brak{\alpha + \frac{\pi}{2}}$, $\cos \brak{\beta + \frac{\pi}{2}}$, $\cos \brak{\gamma + \frac{\pi}{2}}$ is equal to 
        \hfill{\brak{1987 - 2 Marks}}



    \item The value of
        $\sin\frac{\pi}{14}\sin\frac{3\pi}{14}\sin\frac{5\pi}{14}\sin\frac{7\pi}{14}
        \\ \sin\frac{9\pi}{14}\sin\frac{11\pi}{14}\sin\frac{13\pi}{14} $is equal to  

        \hfill{\brak{1991 - 2 Marks}}



    \item If $K = \sin\brak{\frac{\pi}{18}}\sin\brak{\frac{5\pi}{18}}\sin\brak{\frac{7\pi}{18}}$ then the numerical value of $K$ is  
        \hfill{\brak{1993 - 2 Marks}}



    \item If $A > 0, B>0$ and $A + B = \frac{\pi}{3}$, then the maximum value $\tan A \tan B$ is  
        \hfill{\brak{1993 - 2 Marks}}



    \item General value of $\theta$ satisfying the equation $\tan^{2}\theta +\sec2\theta = 1$ is 
        \hfill{\brak{1996 - 1 Mark}}



    \item The real roots of the equation $\cos^{7}x + \sin^{4}x = 1$ in the interval $\brak{-\pi,\pi}$ are 
        \hfill{\brak{1997 - 2 Marks}}

