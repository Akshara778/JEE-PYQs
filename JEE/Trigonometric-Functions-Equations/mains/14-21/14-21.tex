\iffalse
 \title{ASSIGNMENT-1}
 \author{EE24BTECH11008-ASLIN GARVASIS}
 \section{mains}
\fi
% \begin{enumerate}[start=14]
\item The expression $\frac{\tan A}{1-\cot A} +\frac{\cot A}{1-\tan A}$ can be written as:\\

\hfill {(JEE M 2013)}\\
    \begin{enumerate}
    \item $\sin\brak{A}\cos\brak{A}+1$
    \item $\sec\brak{A}\cosec\brak{A}+1$
    \item $\tan\brak{A}+\cot\brak{A}$ 
    \item $\sec\brak{A}+\cosec\brak{A}$
    \end{enumerate}
\item Let $f_{k}x=\frac{1}{k}$ $\brak{\sin^{k}x+\cos^{k}x}$ where $x\in R$ AND $k\geq 1 \cdot $\\
 Then $f_{4}\brak{x}-f_{6}\brak{x}$ equals

\hfill {(JEE M 2014)}\\
    \begin{enumerate}
    \item  $\frac{1}{4}$ 
     \item $\frac{1}{12}$
    \item $\frac{1}{6}$
    \item $\frac{1}{3}$
    \end{enumerate}
\item If $0 \ge x \ge 2\pi$, then the number of real values of x, which \\  satisfy the equation $\cos x+\cos2x+\cos3x+\cos4x=0$ is:

\hfill {(JEE M 2016)}\\
    \begin{enumerate}
    \item $7$
    \item $9$
    \item $3$
    \item $5$
    \end{enumerate}
    
\item If $5${$\tan^2x-\cos^2x=2\cos2x+9$} then value of $\cos 4x$ is:

\hfill{(JEE M 2017)}\\
    \begin{enumerate}
    \item $\frac{-7}{9}$ 
    \item $\frac{-3}{5}$
    \item $\frac{1}{3}$
    \item $\frac{2}{9}$\\
    \end{enumerate}
 \item If sum of all the solutions of the equation\\
  $8 \cos\brak{x} \cdot \cos\brak{\frac{\pi}{6} + x }$ $\cdot \cos \brak{\frac{\pi}{6}}$ - $\frac{1}{2} - 1 \text{ in }$ $\sbrak {0, \pi}$\text{ is } k $\pi$

 then k is equal to:\\
 
\hfill{(JEE M 2018)}\\
\begin{enumerate}
\item $\frac{13}{9}$
\item $\frac{8}{9}$\\
\item  $\frac{20}{9}$
\item  $\frac{2}{3}$\\
\end{enumerate}
  \item For any $\theta \in \brak{\frac{\pi}{4}}$,$\brak{\frac{\pi}{2}}$ the expression\\
 $3\brak{\sin\theta-\cos\theta^4 +6}$ $\brak{\sin\theta+\cos\theta^2 +4\sin^{6}\theta}$ equals:\\
 
\hfill {(JEE M 2019-9 Jan  M)}\\
 \begin{enumerate}
 \item $13-4\cos^2\theta +6\sin^2\theta \cos^2\theta $\\
 \item  $13-4\cos^6\theta$\\
\item  $13-4\cos^2\theta +6\cos^4\theta$\\
 \item $13-4\cos^2\theta +2\sin^2\theta \cos^2\theta$\\
 \end{enumerate}
\item The value of\\ $\cos^210\degree-\cos10\degree\cos50\degree+\cos^250\degree$ is:\\

\hfill {(JEE M 2019-9 April M)}\\
\begin{enumerate}
\item $\frac{3}{4}$ $+\cos20\degree$
\item $\frac{3}{4}$\\
 \item $\frac{3}{2}$ $\brak{1+\cos20\degree}$ 
 \item $\frac{3}{2}$\\
 \end{enumerate}
\item Let S=$\theta \in \sbrak{-2\pi, 2\pi}$ :$2\cos^2\theta + 3\sin\theta=0.$\\
 Then the sum of the elements of S is:\\
 
\hfill {(JEE M 2019-9 April M)}\\
\begin{enumerate}
\item $\frac{13\pi}{6}$ 
\item $\frac{5\pi}{3}$
 \item $2$
 \item $1$\\
\end{enumerate} 
% \end{enumerate}
