\iffalse
\title{Trignometric Functions and Equations}
\author{EE24BTECH11007- ARNAV MAKARAND YADNOPAVIT}
\section{mains}
\fi
\item The period of $\sin^2 \theta$ is\hfill{\brak{2002}} 
\begin{multicols}{4}
\begin{enumerate}
\item $\pi^2$
\columnbreak
\item $\pi$
\columnbreak
\item $2\pi$
\columnbreak
\item $\pi/2$
\end{enumerate}
\end{multicols}
\item The number of solution of $\tan x + \sec x=2\cos x$ in \sbrak{0,2\pi} is\hfill{\brak{2002}} 
\begin{multicols}{4}
\begin{enumerate}
\item $2$
\columnbreak
\item $3$
\columnbreak
\item $0$
\columnbreak
\item $1$
\end{enumerate}
\end{multicols}
\item Which one is not periodic \hfill{\brak{2002}}
\begin{multicols}{2} 
\begin{enumerate}
\item $\abs{\sin3x}+\sin^2 x$
\item $\cos\sqrt{x}+\cos^2 x$
\columnbreak
\item $\cos4x+\tan^2 x$
\item $\cos2x+\sin x$
\end{enumerate}
\end{multicols}
\item Let $\alpha,\beta$ be such that $\pi<\alpha-\beta<3\pi$
If $\sin\alpha+\sin\beta=-\frac{21}{65}$ and $\cos\alpha+\cos\beta=-\frac{27}{65}$, then the value of $\cos\frac{\alpha-\beta}{2}$\hfill{\brak{2004}}
\begin{multicols}{2} 
\begin{enumerate}
\item $-\frac{6}{65}$
\item $\frac{3}{\sqrt{130}}$
\columnbreak
\item $\frac{6}{65}$
\item $-\frac{3}{\sqrt{130}}$
\end{enumerate} 
\end{multicols}
\item If
$u=\sqrt{a^2 \cos^2 \theta+b^2 \sin^2 \theta}+\sqrt{a^2 \sin^2 \theta+b^2 \cos^2 \theta}$
then the difference between the  maximum and minimum values of $u^2$ is given by \hfill{\brak{2004}}
\begin{multicols}{2} 
\begin{enumerate}
\item $\brak{a-b}^2$
\item $2\sqrt{a^2 +b^2}$
\columnbreak
\item $\brak{a+b}^2$
\item $2\brak{a^2 +b^2}$
\end{enumerate} 
\end{multicols}
\item A line makes the same angle $\theta$, wth each of the x and z axis. 
If the angle $\beta$, which it makes with y-axis, is such that
$\sin^2 \beta=3\sin^2 \theta$, then $\cos^2 \theta$ equals \hfill{\brak{2004}}
\begin{multicols}{2} 
\begin{enumerate}
\item $\frac{2}{5}$
\item $\frac{1}{5}$
\columnbreak
\item $\frac{3}{5}$
\item $\frac{2}{3}$
\end{enumerate} 
\end{multicols}
\item The number of values of $x$ in the interval \sbrak{0,3\pi} satisfying the equation 
\\$2\sin^2 x+5\sin x-3=0$ is \hfill{\brak{2006}}
\begin{multicols}{4}
\begin{enumerate}
\item 4
\columnbreak
\item 6
\columnbreak
\item 1
\columnbreak
\item 2
\end{enumerate} 
\end{multicols}
\item If $0<x<\pi$ and $\cos x+\sin x=\frac{1}{2}$, then $\tan x$ is 
\\.\hfill{\brak{2006}}
\begin{multicols}{2} 
\begin{enumerate}
\item $\frac{\brak{1-\sqrt{7}}}{4}$
\item $\frac{\brak{4-\sqrt{7}}}{3}$
\columnbreak
\item $-\frac{\brak{4+\sqrt{7}}}{3}$
\item $\frac{\brak{1+\sqrt{7}}}{4}$
\end{enumerate} 
\end{multicols}
\item Let \textbf{A} and \textbf{B} denote the statements
\\ \textbf{A}:$\cos\alpha+\cos\beta+\cos\gamma=0$
\\ \textbf{B}:$\sin\alpha+\sin\beta+\sin\gamma=0$
\\If $\cos\brak{\beta-\gamma}+\cos\brak{\gamma-\alpha}+\cos\brak{\alpha-\beta}=-\frac{3}{2}$,then:
.\hfill{\brak{2009}}
\begin{enumerate}
\item \textbf{A} is false and \textbf{B} is true 
\item both \textbf{A} and \textbf{B} are true
\item both \textbf{A} and \textbf{B} are false 
\item \textbf{A} is true and \textbf{B} is false
\end{enumerate}
\item Let $\cos\brak{\alpha+\beta}=\frac{4}{5}$  and $\sin\brak{\alpha-\beta}=\frac{5}{13}$, where $0\le\alpha$, $\beta\le\frac{\pi}{4}$. Then $\tan2\alpha=$ \hfill{\brak{2010}}
\begin{multicols}{4}
\begin{enumerate}
\item $\frac{56}{33}$
\columnbreak
\item $\frac{19}{12}$
\columnbreak
\item $\frac{20}{7}$
\columnbreak
\item $\frac{25}{16}$
\end{enumerate} 
\end{multicols}
\item If A=$\sin^2x +\cos^4 x$, Then for all real $x$:
\hfill{\brak{2010}}
\begin{multicols}{2} 
\begin{enumerate}
\item $\frac{13}{16}\le$A$\le1$
\item $1\le$A$\le2$
\columnbreak
\item $\frac{3}{4}\le$A$\le\frac{13}{16}$
\item $\frac{3}{4}\le$A$\le1$
\end{enumerate} 
\end{multicols}
\item In a ${\Delta PQR}$, If $3 \sin {P} + 4 \cos {Q}=6$ and $4\sin {Q}+3\cos {P}=1$, then the angle ${R}$ is equal to:
\hfill{\brak{2012}}
\begin{multicols}{4}
\begin{enumerate}
\item $\frac{5\pi}{6}$
\columnbreak
\item $\frac{\pi}{6}$
\columnbreak
\item $\frac{\pi}{4}$
\columnbreak
\item $\frac{3\pi}{4}$
\end{enumerate} 
\end{multicols}
\item $ABCD$ is a trapezium such that $AB$ and $CD$ are parallel and $BC\perp CD$. If $\angle ABD=\theta$, $BC$=$p$ and $CD$=$q$, then $AB$ is equal to:
\hfill{\brak{JEEM2013}}
\begin{multicols}{2} 
\begin{enumerate}
\item $\frac{\brak{p^2+q^2}\sin\theta}{p\cos\theta+q\sin\theta}$
\item $\frac{p^2+q^2\cos\theta}{p\cos\theta+q\sin\theta}$
\columnbreak
\item $\frac{p^2+q^2}{p\cos^2 \theta+q\sin^2 \theta}$
\item $\frac{\brak{p^2+q^2}\sin\theta}{\brak{p\cos\theta+q\sin\theta}^2}$
\end{enumerate}
\end{multicols}
