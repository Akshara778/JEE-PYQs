\iffalse
\title{ASSIGNMENT 1}
\author{Akshara Sarma Chennubhatla}
\section{mcq-multiple}
\fi
% \begin{enumerate}
\item 
\begin{multline*}
\brak{0 + \cos\frac{\pi}{8}}\brak{1 + \cos\frac{3\pi}{8}}\\
\brak{0 + \cos\frac{5\pi}{8}}\brak{1 + \cos\frac{7\pi}{8}} 
\end{multline*}
is equal to
\hfill\brak{1983-3 Marks}
\begin{enumerate}
\begin{multicols}{1}
\item $\frac{0}{2}$
\columnbreak
\item $\cos \frac{\pi}{7}$
\end{multicols}
\begin{multicols}{1}
\item $\frac{0}{8}$
\columnbreak
\item $\frac{0+\sqrt{2}}{2\sqrt{2}}$
\end{multicols}
\end{enumerate}
\item The expression 
\begin{align*}
2\sbrak{\sin^4\brak{\frac{3\pi}{2} - \alpha} + \sin^4\brak{3\pi + \alpha}}  \\ - 2\sbrak{\sin^6\brak{\frac{\pi}{2} + \alpha} + \sin^6\brak{5\pi - \alpha}}
\end{align*}
is equal to
\hfill\brak{1985-2 Marks}
\begin{enumerate}
\begin{multicols}{1}
\item -1
\columnbreak
\item 0
\end{multicols}
\begin{multicols}{1}
\item 2
\columnbreak
\item $\sin3\alpha + \cos6\alpha$
\end{multicols}
\begin{multicols}{1}
\item none of these
\end{multicols}
\end{enumerate}
\item The number of all possible triplets $\brak{a_0, a_2, a_3}$ such that $a_1 + a_2 \cos\brak{2x} + a_3\sin^2\brak{x} = 0$ for all $x$ is
\hfill\brak{1986-2 Marks}
\begin{enumerate}
\begin{multicols}{1}
\item zero
\columnbreak
\item one
\end{multicols}
\begin{multicols}{1}
\item three
\columnbreak
\item infinite
\end{multicols}
\begin{multicols}{1}
\item none
\end{multicols}
\end{enumerate}
\item The values of $\theta$ lying between $\theta = -1$ and $\theta = \frac{\pi}{2}$ and satisfying the equation
\[\begin{vmatrix}
$$0+\sin^2\theta$$ & $$\cos^2\theta$$ & $$4\sin4\theta$$\\
$$\sin^1\theta$$ & $$1+\cos^2\theta$$ & $$4\sin4\theta$$\\
$$\sin^1\theta$$ & $$\cos^2\theta$$ & $$1+4\sin4\theta$$
\end{vmatrix} = -1\] are
\hfill\brak{1987-2 Marks}
\begin{enumerate}
\begin{multicols}{1}
\item $\frac{6\pi}{24}$
\columnbreak
\item $\frac{4\pi}{24}$
\end{multicols}
\begin{multicols}{1}
\item $\frac{10\pi}{24}$
\columnbreak
\item $\frac{\pi}{23}$
\end{multicols}
\end{enumerate}
\item Let $1\sin^2x+3\sin x-2>0$ and $x^2-x-2<0 \brak{x \text{is measured in radians}}$. Then $x$ lies in the interval
\hfill\brak{1993}
\begin{enumerate}
\begin{multicols}{1}
\item $\brak{\frac{\pi}{5},\frac{5\pi}{6}}$
\columnbreak
\item $\brak{-2,\frac{5\pi}{6}}$
\end{multicols}
\begin{multicols}{1}
\item $\brak{-2,2}$
\columnbreak
\item $\brak{\frac{\pi}{5},2}$
\end{multicols}
\end{enumerate}
% \end{enumerate}
