\iffalse

\title{JEE ADVANCED}
\author{ee24btech11004 - ANKIT JAINAR}
\section{mcq-multiple}
\fi

\item The minimum value of expression $\sin{\alpha} + \sin{\beta} + \sin{\gamma}$, where $\brak{\alpha,\beta,\gamma}$ are real numbers satisfying $\brak{\alpha+\beta+\gamma}=\pi$  is \hfill\brak{1995}
\begin{enumerate}
    \item positive
    \item $0$
    \item negative 
    \item $-3$
\end{enumerate}
\item The number of values of $x$ in the interval $\sbrak{0,5\pi}$ satisfying equation \\$3 \sin{\brak{x^2}}-7 \sin{x+2}=0$  \hfill\brak{1998-2 Marks} 
\begin{enumerate}
    \item $0$
    \item $5$
    \item $6$
    \item $10$
\end{enumerate}

\item Which of the following number\brak{s} is/are rational? \hfill\brak{1998-2 Marks} 
\begin{enumerate}
    \item $\sin{15}\degree$
    \item $\cos{15}\degree$
    \item $\sin{15}\degree \cos{15}\degree$
    \item $\sin{15}\degree \cos{75}\degree$
\end{enumerate}
\item For a positive integer $n$, let \ 
$f_n\brak{\theta} = \brak{\tan\frac{\theta}{2}}\brak{1+\sec{\theta}}\brak{1+\sec{2\theta}}\brak{1+\sec4\theta}\dots\brak{1+\sec2^n{\theta}}.$ \\Then  \hfill\brak{1999 - 3Marks}
\begin{enumerate}
    \item $f_2\brak{\frac{\pi}{16}} = 1$
    \item $f_3\brak{\frac{\pi}{32}} = 1$
    \item $f_4\brak{\frac{\pi}{64}} = 1$
    \item $f_5\brak{\frac{\pi}{128}} = 1$
\end{enumerate}
\item If $\frac{\sin^4{x}}{2}+\frac{\cos^4{x}}{3}=\frac{1}{5}$ , Then \hfill\brak{2009} 
\begin{enumerate}
    \item $\tan^2{x}=\frac{2}{3}$
    \item $\frac{\sin^8{x}}{8}+\frac{\cos^8{x}}{27}=\frac{1}{125}$
    \item $\tan^2{x}=\frac{1}{3}$
    \item $\frac{\sin^8{x}}{8}+\frac{\cos^8{x}}{27}=\frac{2}{125}$
\end{enumerate}
\item For $ 0<\theta <\frac{\pi}{2}$, the solution\brak{s} of $\sum_{m=1}^{6}\cosec{\brak{\theta+\frac{\brak{m-1}\pi}{4}}}\cosec{{\brak{\theta}+\frac{m\pi}{4}} }= 4\sqrt{2}$ is\brak{are} \hfill\brak{2009}
\begin{enumerate}
    \item $\frac{\pi}{4}$
    \item $\frac{\pi}{6}$
    \item $\frac{\pi}{12}$
    \item $\frac{5\pi}{12}$
\end{enumerate}

\item Let $\theta, \varphi \in [0,2\pi]$ be such that $2 \cos\brak{\theta\brak{1-\sin \varphi}}= \sin^2\brak{\theta\brak{\tan\frac{\theta}{2}}+\cot\frac{\theta}{2}}\cos \varphi-1$ ,$\tan\brak{2\pi-\theta}>0$ and $-1<\sin{\theta}<-\frac{\sqrt{3}}{2}$, then $\varphi$ cannot satisfy \hfill\brak{2012}
\begin{enumerate}
    \item $0<\varphi<\frac{\pi}{2}$
    \item $\frac{\pi}{2}<\varphi<\frac{4\pi}{3}$
    \item $\frac{4\pi}{3}<\varphi<\frac{3\pi}{2}$
    \item $\frac{3\pi}{2}<\varphi<2\pi$
\end{enumerate}
\item The number of points in $\brak{-\infty, \infty}$, for which $x - x \sin x - \cos x = 0$, is \hfill\brak{JEE Adv.2013}
\begin{enumerate}
    \item $6$
    \item $4$
    \item $2$
    \item $0$
\end{enumerate}
\item Let $f\brak{x}=x\sin\pi x $, $ x>0 $. Then for all  natural numbers n, \brak{f^{\prime}\brak{x}} vanishes at 
\hfill\brak{JEE Adv. 2013}
\begin{enumerate}
    \item A unique point in the interval $\brak{n,n+\frac{1}{2}}$
    \item A unique point in the interval $\brak{n+\frac{1}{2},n+1}$
    \item A unique point in the interval $\brak{n,n+1}$
    \item Two points in the interval $\brak{n,n+1}$
\end{enumerate}
\item Let $\alpha$ and $\beta$ be non-zero real numbers such that 2$\brak{\cos \beta - \cos \alpha}$+$\cos \alpha \cos \beta=1$.Then which of the following is/are true? \hfill\brak{JEE Adv.2017}
\begin{enumerate}
    \item $\tan{\brak{\frac{\alpha}{2}}+\sqrt{3}\tan\brak{\frac{\beta}{2}}}=0$
    \item $\sqrt{3}\brak{\tan{\frac{\alpha}{2}}}+\tan\brak{{\frac{\beta}{2}}}=0$
    \item $\tan{\brak{\frac{\alpha}{2}}}-\tan{\brak{\frac{\beta}{2}}}=0$
    \item $\sqrt{3}\tan{\brak{\frac{\alpha}{2}}}-\tan{\brak{\frac{\beta}{2}}}=0$
\end{enumerate}





