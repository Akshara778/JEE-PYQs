\iffalse
  \title{Trignometric Functions and Equations}
  \author{EE24BTECH11006- Arnav Mahishi}
  \section{integer}
\fi
\item{The number of all possible values of $\theta $ where   $ 0<\theta<\pi $ for which the system of equations$ \\$
\begin{enumerate}
\item $\brak{y+Z}\cos3\theta=\brak{xyz}\sin3\theta$\\

\item $x\sin3\theta=\frac{2\cos3\theta}{y}+\frac{2\sin3\theta}{z}$\\

\item $\brak{xyz}\sin3\theta=\brak{y+2z}\cos3\theta +y\sin3\theta$\\
\end{enumerate}



have a solution $\brak{x_{o},y_{o},x_{o}}$ with $y_{o}z_{o}$$\neq0$ is \hfill\brak{2010}
\\
}
\item{The number of all possible values of $\theta$ in the interval,
$\brak{\frac{-\pi}{2},\frac{\pi}{2}}$  such that $\theta\neq\frac{n\pi}{5} for n=0,\pm1,\pm2 $ and $\tan\theta=\cot5\theta $ as well as $\sin2\theta=\cos4\theta$  is \hfill{\sbrak{2010}}
}\\

\item{The maximum value of the expression
\\$\frac{1}{\sin^2\theta+3\sin\theta \cos\theta+5\cos^2\theta}$ is  \hfill{\sbrak{2010}}
\\}


\item{Two parallel chords of a circle of radius $2$ are at a distance$\brak{\sqrt{3}+1} $\space apart. If the chords subtend at the center, angles of $\frac{\pi}{k} and \frac{2\pi}{k}$, where $k>0$, the value of \sbrak{k} is \hfill{\sbrak{2010}}}
\\

\item{The positive integer value of $n>3$ satisfying the equation $\frac{1}{sin\brak{\frac{\pi}{n}}}=\frac{1}{sin\brak{\frac{2\pi}{n}}}+\frac{1}{sin\brak{\frac{3\pi}{n}}}$ is\hfill{\sbrak{2010}}}


