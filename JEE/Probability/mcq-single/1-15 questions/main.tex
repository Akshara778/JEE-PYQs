
\iffalse
  \title{PROBABILITY}
  \author{Teja vardhan}
  \section{mcq-single}
\fi
%   \begin{enumerate}
\item A problem in mathematics is given to three students A, B, C and their respective probability of solving the problem is $\frac{1}{2}, \frac{1}{3} and \frac{1}{4}$. Probability that the problem is solved is  \hfill [2002]

        \begin{multicols}{4}
        \begin{enumerate}
            \item $\frac{3}{4}$
            \item $\frac{1}{2}$
            \item $\frac{2}{3}$
            \item $\frac{1}{3}$   
        \end{enumerate}
        \end{multicols}
        


    \item A and B are events such that \pr{A \cup B} = $\frac{3}{4}$, \pr{A \cup B} = $\frac{1}{4}$, \pr{\overline{A}} = $\frac{2}{3}$ then \pr{\overline{A} \cup B} \hfill [2002]

        \begin{multicols}{4}
        \begin{enumerate}
            \item $\frac{3}{4}$
            \item $\frac{1}{2}$
            \item $\frac{2}{3}$
            \item $\frac{1}{3}$
        \end{enumerate}
        \end{multicols}


    \item A dice is tossed 5 times. Getting an odd number is considered a success. Then the variance of distribution of success \hfill [2002]

        \begin{multicols}{4}
        \begin{enumerate}
            \item $\frac{8}{3}$
            \item $\frac{3}{8}$
            \item $\frac{5}{8}$
            \item $\frac{1}{4}$   
        \end{enumerate}
        \end{multicols}
        

    \item The mean and variance of a random variable X having binomial distribution are 4 and 2 respectively, then \pr{x = 1} is \hfill [2003]

        \begin{multicols}{4}
        \begin{enumerate}
            \item $\frac{1}{4}$
            \item $\frac{1}{32}$
            \item $\frac{1}{16}$
            \item $\frac{1}{8}$   
        \end{enumerate}
        \end{multicols}

        
    \item Events A, B, C are mutually exclusive events such that \pr{A}= $\frac{3x +1}{3}$, $\pr{B} = \frac{1 - x}{4}$ and \pr{C} = $\frac{1 - 2x}{2}$ The set of possible values of x are in the interval. \hfill [2003]

        \begin{multicols}{4}
        \begin{enumerate}
            \item $[0 , 1]$
            \item $[\frac{1}{3} , \frac{1}{2}]$
            \item $[\frac{1}{3} , \frac{2}{3}]$
            \item $[\frac{1}{3} , \frac{13}{8}]$
        \end{enumerate}
        \end{multicols}


    \item Five horses are in a race. Mr.A selects two of the horses at random and bets on them. The probability that Mr.A selected the winning horse is  \hfill [2003]

        \begin{multicols}{4}
        \begin{enumerate}
            \item $\frac{2}{5}$
            \item $\frac{4}{5}$
            \item $\frac{3}{5}$
            \item $\frac{1}{5}$   
        \end{enumerate}
        \end{multicols}


    \item The probability that A speaks truth is $\frac{4}{5}$, while the probability for B is $\frac{3}{4}$. The probability that they contradict each other when asked to speak on a fact is  \hfill [2004]

        \begin{multicols}{4}
        \begin{enumerate}
            \item $\frac{4}{5}$
            \item $\frac{1}{5}$
            \item $\frac{7}{20}$
            \item $\frac{3}{20}$   
        \end{enumerate}
        \end{multicols}

        
    \item A random variable X has the probability distribution: For the events E = {X is a prime number } and F = (X \textless  4), Pr(\(E \cup F\)) is \hfill [2004]

        \begin{table}[h!]
        \centering
        \begin{tabular}{|c|c|c|c|c|c|c|c|c|}
        \hline
            X: &1&2&3&4&5&6&7&8 \\
            \hline
        P(X):&0.2&0.2&0.1&0.1&0.2&0.1&0.1&0.1\\
        \hline
        \end{tabular}
            \end{table}


        \begin{multicols}{4}
        \begin{enumerate}
            \item $0.50$
            \item $0.77$
            \item $0.35$
            \item $0.87$   
        \end{enumerate}
        \end{multicols}


    \item The mean and the variance of a binomial distribution are 4 and 2 respectively. Then the probability of 2 successes is   \hfill [2004]

        \begin{multicols}{4}
        \begin{enumerate}
            \item $\frac{28}{256}$
            \item $\frac{219}{256}$
            \item $\frac{128}{256}$
            \item $\frac{37}{256}$   
        \end{enumerate}
        \end{multicols}


            \item Three houses are available in a locality. Three persons apply for the houses. Each applies for one house without consulting others. The probability that all the three apply for the same house is  \hfill [2005]

        \begin{multicols}{4}
        \begin{enumerate}
            \item $\frac{2}{9}$
            \item $\frac{1}{9}$
            \item $\frac{8}{9}$
            \item $\frac{7}{9}$   
        \end{enumerate}
        \end{multicols}



            \item A random variable x has Poisson distribution with mean 2. Then \pr{X > 1.5} equals \hfill [2005]

        \begin{multicols}{4}
        \begin{enumerate}
            \item $\frac{2}{e^2}$
            \item $0$
            \item $1 - \frac{3}{e^2}$
            \item $\frac{3}{e^2}$   
        \end{enumerate}
        \end{multicols}

        
            \item Let A and B be the two events such that $\pr{\overline{A \cup B}} = 1/6$, $\pr{A \cup B} = 1/4$ and $\pr{\overline{A}} = 1/4$, where $\overline{A}$ stands for complement of event A. Then events A and B are     \hfill [2005]

        
        \begin{enumerate}
            \item equally likely and mutually exclusive
            \item equally likely but not independent 
            \item independent but not equally likely
            \item mutually exclusive and independent 
        \end{enumerate}
        




            \item At a telephone enquiry system the number of phone cells regarding relevent enquiry follow Possion distribution with an average of 5 phone calls during 10 minute time intervals. The probability that there is at the most one phone call during a 10 - minute time period \hfill [2006]

        \begin{multicols}{4}
        \begin{enumerate}
            \item $\frac{6}{5^e}$
            \item $\frac{5}{6}$
            \item $\frac{6}{55}$
            \item $\frac{6}{e^5}$   
        \end{enumerate}
        \end{multicols}


         \item Two aeroplanes I and II bomb a target in succession. The probabilities of I and II scoring a hit correctly are 0.3 and 0.2, respectively. The second plane will bomb only if the first misses the target. The probability that the target is hit by the second plane is      \hfill [2007]

        \begin{multicols}{4}
        \begin{enumerate}
            \item 0.2
            \item 0.7 
            \item 0.06
            \item 0.14 
        \end{enumerate}
        \end{multicols}

    \item A pair of fair dice is thrown independently three times. The probability of getting a score of exactly 9 twice is  \hfill [2007]


    \begin{multicols}{4}
        \begin{enumerate}
            \item $\frac{8}{729}$
            \item $\frac{8}{243}$
            \item $\frac{1}{729}$
            \item $\frac{8}{9}$   
        \end{enumerate}
        \end{multicols}
% \end{enumerate}


